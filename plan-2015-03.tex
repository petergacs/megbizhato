\documentclass[12pt]{memoir}

\pagestyle{plain}
%\pagestyle{simple}
\setlrmargins{*}{*}{1}
\checkandfixthelayout

\setcounter{tocdepth}{2}
\setcounter{secnumdepth}{2}
\counterwithout{section}{chapter}
\counterwithin{equation}{section}
%\numberwithin{equation}{section} % in amsmath
%\counterwithout{figure}{chapter}
\counterwithin{figure}{section}

\makeatletter
% To correct a memoir bug:
\renewcommand{\@memmain@floats}{%
  \counterwithin{figure}{section}
  \counterwithin{table}{section}}
\makeatother

\firmlists

% Not with mathdesign.  Before hyperref, otherwise AucTex is in trouble:
\usepackage{amssymb}

% If you do not want the bibliography on a separate page:
\renewcommand{\bibsection}{% 
\section*{\bibname} 
\prebibhook}

\usepackage[backref,hyperindex,colorlinks,linkcolor=blue,citecolor=blue]{hyperref}
\usepackage[numbered]{bookmark} % Allows to place a bookmark, see the title. Shows section numbers.
% \usepackage[all]{hypcap} % After hyperref, to anchor floats correctly.
% \usepackage{float}
 % After hyperref:
\usepackage[algo2e,algosection,tworuled,noend,noline]{algorithm2e}
\usepackage[sf]{gacs} % Process with XeTeX
\usepackage{gacs-algo} % After hyperref.
% After gacs.sty

%\usepackage[pagecolor={Honeydew1}]{pagecolor}

\hyphenation{com-plex-ity des-tin-at-ion co-lon-ies}

\newcommand{\shownotes}{1}
\ifnum\shownotes=1
\newcommand{\authnote}[3]
{\text{{ \textcolor{#3}{\( \langle\hspace{-0.2em}\langle \)\textsf{\footnotesize #1: #2}\( \rangle\hspace{-0.2em}\rangle \)}}}}
\else
\newcommand{\authnote}[2]{}
\fi
\newcommand{\Pnote}[1]{{\authnote{P}{#1}{cyan}}}
\newcommand{\Inote}[1]{{\authnote{I}{#1}{blue}}}

\renewcommand{\le}{\leq}
\renewcommand{\ge}{\geq}

\newcommand{\fld}[1]{\ensuremath{\textit{#1\/}}}
\newcommand{\rul}[1]{\ensuremath{\texttt{\slshape #1\/}}}
\newcommand{\maj}{\mathrm{maj}}
\newcommand{\sign}{\mathop\mathrm{sign}}

\newcommand{\tEnd}{f_{\mathrm{end}}}
\newcommand{\tZig}{f_{\mathrm{zig}}}
\newcommand{\tHeal}{f_{\mathrm{heal}}}
\newcommand{\tRebuild}{f_{\mathrm{rebuild}}}

% Using def for the possibility of switching between LaTeX and XeTeX:
\def\B{B}  
\def\U{U}

\newcommand{\Bad}{\mathit{Bad}}
\newcommand{\Vacant}{\mathit{Vac}}
\newcommand{\blank}{\text{\textvisiblespace}}
\newcommand{\Configs}{\mathrm{Configs}}
\newcommand{\D}{D}
\newcommand{\E}{E}
\def\G{G}
\newcommand{\h}{h}
\renewcommand{\H}{H}
\newcommand{\hc}{\tilde h}
\newcommand{\Noise}{\mathit{Noise}}
\newcommand{\Output}{\mathit{output}}
\newcommand{\PenetrationLen}{\mathrm{PenetLen}}
\newcommand{\Plus}{\oplus}
\newcommand{\Minus}{\ominus}
\newcommand{\pos}{\mathrm{pos}}
\newcommand{\score}{\mathrm{score}}
\newcommand{\R}{R}
\newcommand{\Tu}{T}
\newcommand{\Tus}{T^{*}}
\renewcommand{\V}{V}
\newcommand{\F}{F}
\newcommand{\Z}{Z}
\newcommand{\z}{z}


\newcommand{\Addr}{\fld{Addr}}
\newcommand{\cAddr}{\fld{cAddr}}
\newcommand{\cCanTurn}{\fld{cCanTurn}}
\newcommand{\Core}{\fld{Core}}
\newcommand{\cCore}{\fld{cCore}}
\newcommand{\Dir}{\fld{Dir}}
\newcommand{\cDir}{\fld{cDir}}
\newcommand{\Drift}{\fld{Drift}}
\newcommand{\Doomed}{\fld{Doomed}}
\newcommand{\cDrift}{\fld{cDrift}}
%\renewcommand{\G}{\fld{NonAdj}}
\newcommand{\NonAdj}{\fld{NonAdj}}
\newcommand{\cHold}{\fld{cHold}}
\newcommand{\Index}{\fld{Index}}
\newcommand{\Info}{\fld{Info}}
\newcommand{\cInfo}{\fld{cInfo}}
\newcommand{\Kind}{\fld{Kind}}
\newcommand{\cKind}{\fld{cKind}}
\newcommand{\cLevel}{\fld{cLevel}}
\newcommand{\Mode}{\fld{Mode}}
\newcommand{\cProg}{\fld{cProg}}
\newcommand{\Heal}{\fld{Heal}}
\newcommand{\rHeal}{\rul{Heal}}
\newcommand{\cHeal}{\fld{cHeal}}
\newcommand{\Plan}{\fld{Plan}}
\newcommand{\Rebd}{\fld{Rebd}}
\newcommand{\cRebd}{\fld{cRebd}}
\newcommand{\Stage}{\fld{Stage}}
\newcommand{\State}{\fld{State}}
\newcommand{\cState}{\fld{cState}}
\newcommand{\Sweep}{\fld{Sw}}
\newcommand{\cSweep}{\fld{cSw}}
\newcommand{\cWork}{\fld{cWork}}
\newcommand{\ZigDepth}{\fld{ZigDepth}}
\newcommand{\ZigDir}{\fld{ZigDir}}

%\newcommand{\Bridge}{\mathrm{Bridge}}
\newcommand{\Coordinated}{\mathrm{Coordinated}}
\newcommand{\decode}{\mathrm{decode}}
\newcommand{\dir}{\mathrm{dir}}
\newcommand{\encode}{\mathrm{encode}}
\newcommand{\front}{\mathrm{front}}
\newcommand{\Rebuilding}{\mathrm{Rebuilding}}
\newcommand{\Histories}{\mathrm{Histories}}
\newcommand{\Last}{\mathrm{Last}}
\newcommand{\Marking}{\mathrm{Marking}}
\newcommand{\Member}{\mathrm{Member}}
\newcommand{\Committing}{\mathrm{Committing}}
\newcommand{\patch}{\mathrm{patch}}
\newcommand{\Planning}{\mathrm{Planing}}
\newcommand{\Target}{\mathrm{Target}}
\newcommand{\Normal}{\mathrm{Normal}}

\newcommand{\PadLen}{\mathit{PadLen}}
\newcommand{\Interpr}{\mathit{Interpr}}

\newcommand{\Healing}{\mathrm{Healing}}
\newcommand{\start}{\mathrm{start}}
\newcommand{\state}{\mathrm{state}}
\newcommand{\Stem}{\mathrm{Stem}}
\newcommand{\tape}{\mathrm{tape}}
\newcommand{\TransferSw}{\mathrm{TransferSw}}
\newcommand{\Un}{\mathrm{Univ}}

\newcommand{\increment}[1]{#1\mathord{+}\mathord{+}}
\newcommand{\decrement}[1]{#1\mathord{-}\mathord{-}}


\newcommand{\ruAddrJmp}{\rul{AddrJmp}}
\newcommand{\Alarm}{\rul{Alarm}}
% \newcommand{\Commit}{\rul{Commit}}
\newcommand{\Comp}{\rul{Compute}}
\newcommand{\BigAlarm}{\rul{BigAlarm}}
%\newcommand{\Vacate}{\rul{Vacate}}
% \newcommand{\Mark}{\rul{Mark}}
\newcommand{\Move}{\rul{Move}}
% \newcommand{\Plan}{\rul{Plan}}
\newcommand{\ruSwing}{\rul{Swing}}
\newcommand{\Transfer}{\rul{Transfer}}
\newcommand{\UsefulComp}{\rul{UsefulComp}}
\newcommand{\WriteRulesBit}{\rul{WriteRulesBit}}
\newcommand{\Zigzag}{\rul{Zigzag}}

\newcommand{\mrk}{\mathrm{mrk}}
\newcommand{\K}{K}
\newcommand{\loc}{\ell_\mrk}
\newcommand{\N}{\mathbf{N}}
\newcommand{\Zg}{\mathcal{Z}_g}

\newcommand{\Cns}[2]{#1_{\textrm{\upshape #2}}}
\newcommand{\cns}[1]{\Cns{c}{#1}}
\newcommand{\cEsc}{\cns{escape}}
\newcommand{\cCleanS}{\cns{clean-s}}
\newcommand{\cCleanT}{\cns{clean-t}}
\newcommand{\cCDepth}[1]{\cns{c-depth-#1}}
\newcommand{\cSpill}{\cns{spill}}
\newcommand{\cSpace}{\cns{space}}
\newcommand{\cDecr}{\cns{decr}}
\newcommand{\cIncr}{\cns{incr}}
\newcommand{\mPlainEdge}{\Cns{\mu}{plain-edge}}
\newcommand{\mHalfTurn}{\Cns{\mu}{half-turn}}
\newcommand{\mTurnDone}{\Cns{\mu}{turn-done}}
\newcommand{\mRebuild}{\Cns{\mu}{rebuild}}

\begin{document}

\title{Notes to Ilir on the Turing machine}
% Why do I need this?  Some people get the title bookmarked even without this.
\bookmark[page=1,level=0]{Notes to Ilir on the Turing machine}

\author{P\'eter}

\maketitle

\section{The problem}

I am returning to the seemingly innocent problem of proving the Spill Bound property
in induction, which I am finding much nastier than originally imagined.
I recall that I missed the following problem: 
the head slides away very far on the dirt, 
comes back \( \Tu^{*} \) time later to leave an island in a burst, 
then repeats this many times.
This way, the dirt seems to be capable of spilling out without bound.
The solution I proposed in an earlier letter was to add a parameter \( \lambda \)
corresponding roughly to the level, require \( Q \) to be many times bigger than \( \lambda \),
and add an axiom that passing over an area \( \lambda \) times will clean it.
However, I was not able to apply this in induction, and am proposing now a significantly
more complex model.
Here, still many details must be worked out, but I feel that this has more chance of succeeding.

\section{Proposed new axioms}

The new axioms are getting too complex.
Maybe one can announce only part of them at the beginning, enough to be
able to prove the bursty case, and postpone the rest of them later.

Let us have a parameter \( \lambda \) (for ``level'').
There is also a new property of positions, called \df{level}, and written \( L(x) \),
taking values between 0 and \( \lambda \).
Clean cells will be the ones with level \( \lambda \).
There will be also some non-integer values: the boundary area of a clean interval will
be an interval of size between \( \B \) and \( 3\B \) and will have a constant
value of \( L(x) \) equal to one of the values
\begin{align*}
 \lambda-\mu_{i}
 \end{align*}
for some constants \( 0<\mu_{i}<1 \), where
\begin{align*}
     0 &<\mRebuild<\mTurnDone<\mHalfTurn<\mPlainEdge<1.
 \end{align*}
This area will be called a \df{buffer}, and could be called \df{plain} buffer, a \df{half-turn} buffer, a
\df{full-turn} buffer, and a \df{rebuilding} buffer.
Their role will be understood in the definition of the simulation code.

The \df{score} \( \score(I) \) over an interval is defined as
 \begin{align}\label{eq:score}
  \score(I)=\sum_{x\in I}2^{L(x)}.
 \end{align}

The following is required, which will make sure that every passing of the right
buffer of a clean interval increases the score of any large interval around it.

\begin{enumerate}[label=\upshape{(T\arabic*)}, ref=T\arabic*]
\item\label{i:trans.to-half-turn} If a plain buffer is passed over from left
then it becomes either a half-turn or a rebuilding buffer.

\item If a rebuilding buffer is passed over rom right then
it turns into a plain buffer, with its left end moved right by at least \( \B \).

\item If the head passes a half-turn from right or turns back from a plain buffer
to the left without passing it, then the buffer becomes
full-turn or plain; in the latter case the clean area advances by at least
\( \B \) to the right.

\item If a full-turn buffer is passed by the head from left then it
becomes a rebuilding or plain buffer; 
in the latter case the clean area advances by at least \( \B \) to the right.

\end{enumerate}

For~\eqref{i:trans.to-half-turn}, we must make sure in simulation of a transition at
cell \( x \) of \( M^{*} \),
that the head will not be captured at \( x+\B^{*} \) during the computation of the transition
that makes it turn back to the left.
To achieve this, the last turn in the simulation will go (significantly)
closer to the colony end \( x+\B^{*} \) than the previous turns.
This way, capturing followed by a return that does not trigger a rebuild
will only be possible if the computation of the simulated transition in the colony of \( x \)
terminates first.
If a rebuild is triggered then a rebuilding buffer is obtained, which is one of the possibilities
in~\eqref{i:trans.to-half-turn}.

I introduced a new constant \( \gamma \) 
in the new definition of sparsity, see the long paper.
Its role is to require that bursts at level \( k \) be
not just one colony but \( \gamma \) colonies separated from each other.

We will fix some  constants \( c_{i} \).
Consider a noise-free time interval \( J \), and a space interval \( I \).

\begin{description}
\item[Spill Bound] 
The old Spill Bound axiom bounded the amount by which the dirt can spill out
of \( I \) during \( J \).
I may have to replace this with a weaker one, saying
that it can be spilled out by at most an amount of \( \beta |I|/\gamma \).

\item[Dwell Cleaning]
The old Dwell Cleaning axiom says the following.
If \( |I| \le \cSpace\B \), and the head spends a cumulative time  \( \Tu \)  during \( J \)
in \( I \) (possibly while entering and exiting repeatedly),
then at time \( v \), a clean point appears in \( I \).
The new one will be similar, but requiring the head to spend the time continuously in \( J \),
not only cumulatively.

% The new one achieves the effect already if the time spent was \( \Tu/G(\lambda) \).
% (The Rebuild procedure takes less time than a work period.)

\item[Pass Cleaning]
Let \( p(t) \) be the head position at time \( t \).
Suppose \( L(p(t))<\lambda \) for all \( t \) in \( J \), moreover
during the time interval \( J \) the head covers the whole of \( I \).
We will have appropriate constants 
\begin{align*}
   0<\cDecr<\cIncr.
 \end{align*}
The score\( \score(I) \) increases by at least 
\begin{align*}
 \cIncr(1+2^{-\lambda})|I|-\cDecr\B .
 \end{align*}
The subtracted term needs to be present only if the head starts but does not end
on a clean position.

\item[Attack Cleaning]
This remains essentially the same.
Maybe it can be  simplified, not requiring the attack, if cleanness of a cell of \( M^{*} \) is
defined to include the health of the neighbor colonies.
\end{description}


\section{Sketch of the inductive proof}

In the inductive construction, we will set \( Q=2^{\lambda} \), \( \lambda^{*}=\lambda+1 \).
For the hierarchy this gives \( Q_{k}=2^{k} \), which is still OK for the sparsity argument.

\subsection{Definition of level}

The level \( L(x) \) should essentially be defined as expected.
Assume that \( x \) is in direction \( j \) from the head.
It has level \( \lambda+1 \)
if it is in a healthy area containing a colony with at least a whole colony to the left and to the right
of \( x \).
% If it only contains a colony in the direction \( j \) then it has level \( \lambda+1/2 \).
% There may come some finer distinctions here.
% The goal is to amortize potential loss or no gain at the time we leave a clean
% area, towards the gain at Exit Cleaning when entering again.


\subsection{Dwell Cleaning}

The proof of the Dwell Cleaning property should go similarly to before.


\subsection{Pass Cleaning}

There are different cases  to consider, but in order to understand the choice of the
score function~\eqref{eq:score}, the following example is characteristic.
Let \( I \) consist of two adjacent intervals \( I_{1} \) and \(  I_{2} \) where  
where \( L(x)<\lambda \) on \( I_{1} \) and \( =\lambda \) on \( I_{2} \), with
\( |I_{2}|\le Q\B \).
Suppose that the head passes through \( I \), first through \( I_{1} \) then \( I_{2} \).
The average score increases by \( \cIncr(1+2^{-(\lambda-1)}) \) on \( I_{1} \).
The problem is that it may not increase \( I_{2} \).
If \( |I_{2}| \) was significantly larger than \( Q\B \) then some colonies will be
found/created on it, raising the average level to \( \lambda+1 \).
The axiom needs only to consider the time interval before this happens.
When it is not happening, we rely on the Exit Cleaning axiom: the interval \( I_{2} \)
will increase by \( \B \), thus the average score will increase by at least \( 2^{\lambda-1} \)
from \( 2^{\lambda-1} \) to \( 2^{\lambda} \) on an interval of size \( \B \).
Since \( Q=2^{\lambda} \), we increase the average score of \( I_{2} \) by \( \gem 1 \).

The general case is messier, but the example shown is motivating the choice of \( 2^{L(x)} \) in
the score function.
We want the amount gained by Exit Cleaning to be
comparable with the amount lost by a burst.
At Exit Cleaning we gain \( 2^{\lambda-1}\B \); even on a clean area of size \( Q\B \)
this increases the average score by \( \gem 1 \).
At a burst the level of an area of size \( \beta\B \) may go from \( \lambda \) to 0,
losing \( 2^{\lambda}\beta\B \).
However, given that no new colony is found/created, bursts are as far apart as 
\( \gamma Q\B =\gamma 2^{\lambda}\B \), so they decrease the average score 
only by \( \beta/\gamma\ll 1 \).

The role of the factor \( 1+2^{-\lambda} \) in the Pass Cleaning axiom is that
even when we only apply the axiom in recursion, the bursts will take away
a fraction \( 2^{-\lambda} \), and
\begin{align*}
 1+2^{-(\lambda-1)}-2^{-\lambda} = 1 + 2^{-\lambda}.
 \end{align*}

% Let us subdivide the time interval \( J \) by the bursts that happen during it,
% into burst-free time intervals \( J_{1},J_{2},\dots \).
% Let \( I_{1}, I_{2},\dots \) be the space intervals between these bursts.
% We will show that for each \( r=1,2,\dots \), the score increases during \( J_{r} \) by
% at least \( \cIncr(1+2^{-\lambda})|I_{r}| \).

% If \( |I_{r}|\le \cSpace Q\B \) then this will come from the Dwell Cleaning axiom
% for \( M^{*} \).

% Suppose \( |I_{r}|>\cSpace Q\B \).
% If \( L(x)<\lambda \) over \( I_{r} \) then this increase comes by induction, by an amount
% \( \cIncr(1+2^{-(\lambda-1)})|I_{r}| \).

% Suppose that \( L(x)=\lambda \) in some subintervals of \( I_{r} \).


\subsection{Spill bound}

Consider the neighboring intervals \( I,K \) separated by point \( x \),
where \( K \) is clean for \( M^{*} \).
We want to limit the amount that \( I \) is going to spill over to \( K \) during a time
interval that the head spends in \( I \) and is free from \( M^{*} \)-noise.

Essentially dirt could only spill into \( K \) via bursts near the boundary \( x \).
Between these bursts, the head would have to pass an interval of size 
\( \ge \gamma\B^{*} =\gamma 2^{\lambda}\B \), and to increase the score of \( I \) 
by \( \gem \gamma\B^{*} \), while the burst decreased it only by 
\( \lem \beta 2^{\lambda}\B\lem\beta\B^{*}\ll \gamma\B^{*} \).
The total number of such times before the score of \( I \) reaches \( 2^{\lambda+1} \) is
\begin{align*}
   \lem 2^{\lambda+1}|I|/\gamma\B^{*} = 2(|I|/\B^{*}) Q/\gamma.
 \end{align*}
The size of the spill is about \( \beta\B \) times bigger than this,
that is \( \lem \beta |I|/\gamma \).

\end{document}
