\documentclass[12pt]{memoir}

\pagestyle{simple}
\setlrmargins{*}{*}{1}
\checkandfixthelayout

\setcounter{tocdepth}{2}
\setcounter{secnumdepth}{2}
\counterwithout{section}{chapter}
%\counterwithout{figure}{chapter}

\makeatletter
% To correct a memoir bug:
\renewcommand{\@memmain@floats}{%
  \counterwithin{figure}{section}
  \counterwithin{table}{section}}
\makeatother

\firmlists

% Not with mathdesign.  Before hyperref, otherwise AucTex is in trouble:
\usepackage{amssymb}

\usepackage{enumitem}
\setlist[enumerate]{leftmargin=*}
\setlist[itemize]{leftmargin=*}
\setlist[description]{font=\mdseries\textsf, leftmargin=1.5em}
%\usepackage[backref,hyperindex,colorlinks,citecolor=blue,pdftex,debug]{hyperref}
\usepackage[backref,hyperindex,colorlinks,citecolor=blue]{hyperref}

% If you do not want the bibliography on a separate page:
\renewcommand{\bibsection}{% 
\section*{\bibname} 
\prebibhook}

\usepackage[all]{hypcap} % After hyperref, to anchor floats correctly.

\usepackage{float}
 % After hyperref:
\usepackage[algo2e,algosection,tworuled,noend,noline]{algorithm2e}

%\usepackage{cheatpf} % After hyperref ?
%\pfshortnumbers{4} % Use short step numbers for all levels >= 3

\usepackage{gacs} % After hyperref.
%\usepackage[sf]{gacs} % After hyperref.
%\usepackage{gacs-algo} % After hyperref.

% After gacs.sty
%\numberwithin{equation}{section} % in amsmath

\myLibertine % defined in gacs.sty

% \myTimes % defined in gacs.sty
%\usepackage[urw-garamond]{mathdesign}
%\usepackage[charter]{mathdesign}
%\myCentury 
%\usepackage[noBBpl]{mathpazo}
\hyphenation{com-plex-ity des-tin-at-ion}

\begin{document}

\title{Sketches}

\author{Peter G\'acs}
% \\ Boston University
% \\ gacs@bu.edu

\maketitle
\thispagestyle{empty}

The Turing machine can only step left and right, cannot stay in place.

When turning back from an end of colony, the head does not come closer than \( 2\E \) to 
the end.
This way, even if it encounters damage, it does not get into the next colony.

How many islands can there be in a colony?
Suppose that it was first visited from the right, and then left on the left.

In visit 1, island 1 can be created and could stay.

After visit 2, island 2 can be created.
Island 1 can stay oly if it was close to the right end.

After visit 3, island 3 can be created.
Now the drift is to the right, so all 3 islands must be erased.








\end{document}
%%% Local Variables: 
%%% mode: latex
%%% TeX-master: t
%%% End: 
