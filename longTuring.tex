\documentclass[12pt]{memoir}

\pagestyle{simple}
\setlrmargins{*}{*}{1}
\checkandfixthelayout

\setcounter{tocdepth}{2}
\setcounter{secnumdepth}{2}
\counterwithout{section}{chapter}
%\counterwithout{figure}{chapter}

\makeatletter
% To correct a memoir bug:
\renewcommand{\@memmain@floats}{%
  \counterwithin{figure}{section}
  \counterwithin{table}{section}}
\makeatother

\firmlists

% Not with mathdesign.  Before hyperref, otherwise AucTex is in trouble:
% \usepackage{amssymb}

\usepackage[backref,hyperindex,colorlinks,citecolor=blue]{hyperref}

% If you do not want the bibliography on a separate page:
\renewcommand{\bibsection}{% 
\section*{\bibname} 
\prebibhook}

\usepackage[all]{hypcap} % After hyperref, to anchor floats correctly.

\usepackage{float}
 % After hyperref:
\usepackage[algo2e,algosection,tworuled,noend,noline]{algorithm2e}

\usepackage{cheatpf} % After hyperref ?
\usepackage{gacs,gacs-algo} % After hyperref.

% After gacs.sty
%\numberwithin{equation}{section} % in amsmath

\myLibertine % defined in gacs.sty

\hyphenation{com-plex-ity des-tin-at-ion}

\newcommand{\fld}[1]{\ensuremath{\textit{#1}}}
\newcommand{\rul}[1]{\ensuremath{\texttt{\slshape #1\/}}}
\newcommand{\maj}{\mathrm{maj}}
\newcommand{\sign}{\mathop\mathrm{sign}}

\renewcommand{\B}{B}
\newcommand{\Bad}{\mathit{Bad}}
\newcommand{\blank}{\text{\textvisiblespace}}
\newcommand{\Configs}{\mathrm{Configs}}
\newcommand{\Damage}{Damage}
\newcommand{\h}{h}
\newcommand{\hc}{\tilde h}
\newcommand{\Noise}{\mathit{Noise}}
\newcommand{\Output}{\mathit{output}}
\newcommand{\pos}{\mathrm{pos}}
\newcommand{\Tu}{T}
\newcommand{\Tus}{T^{*}}
\renewcommand{\U}{U} % What was it?
\newcommand{\Z}{Z}


\newcommand{\Addr}{\fld{Addr}}
\newcommand{\cAddr}{\fld{cAddr}}
\newcommand{\Core}{\fld{Core}}
\newcommand{\cCore}{\fld{cCore}}
\newcommand{\Dir}{\fld{Dir}}
\newcommand{\cDir}{\fld{cDir}}
\newcommand{\Drift}{\fld{Drift}}
\newcommand{\Doomed}{\fld{Doomed}}
\newcommand{\cDrift}{\fld{cDrift}}
\renewcommand{\G}{\fld{NonAdj}} % What was it?
\newcommand{\cHold}{\fld{cHold}}
\newcommand{\Index}{\fld{Index}}
\newcommand{\Info}{\fld{Info}}
\newcommand{\cInfo}{\fld{cInfo}}
\newcommand{\Kind}{\fld{Kind}}
\newcommand{\cKind}{\fld{cKind}}
\newcommand{\cLevel}{\fld{cLevel}}
\newcommand{\Mode}{\fld{Mode}}
\newcommand{\cProg}{\fld{cProg}}
\newcommand{\State}{\fld{State}}
\newcommand{\cState}{\fld{cState}}
\newcommand{\Sweep}{\fld{Sw}}
\newcommand{\cSweep}{\fld{cSw}}
\newcommand{\cWork}{\fld{cWork}}
\newcommand{\ZigDepth}{\fld{ZigDepth}}
\newcommand{\ZigDir}{\fld{ZigDir}}

\newcommand{\Bridge}{\mathrm{Bridge}}
\newcommand{\Coordinated}{\mathrm{Coordinated}}
\newcommand{\decode}{\mathrm{decode}}
\newcommand{\dir}{\mathrm{dir}}
\newcommand{\encode}{\mathrm{encode}}
\newcommand{\Healing}{\mathrm{Healing}}
\newcommand{\Histories}{\mathrm{Histories}}
\newcommand{\Last}{\mathrm{Last}}
\newcommand{\Member}{\mathrm{Member}}
\newcommand{\Neighbor}{\mathrm{Neighbor}}
\newcommand{\Normal}{\mathrm{Normal}}

\newcommand{\PadLen}{\mathit{PadLen}}
\newcommand{\Interpr}{\mathit{Interpr}}

\newcommand{\Recovering}{\mathrm{Recovering}}
\newcommand{\start}{\mathrm{start}}
\newcommand{\state}{\mathrm{state}}
\newcommand{\Stem}{\mathrm{Stem}}
\newcommand{\tape}{\mathrm{tape}}
\newcommand{\TransferSw}{\mathrm{TransferSw}}
\newcommand{\Un}{\mathrm{Univ}}
\newcommand{\Vacant}{\mathrm{Vac}}

\newcommand{\increment}[1]{#1\mathord{+}\mathord{+}}
\newcommand{\decrement}[1]{#1\mathord{-}\mathord{-}}


\newcommand{\ruAddrJmp}{\rul{AddrJmp}}
\newcommand{\Alarm}{\rul{Alarm}}
\newcommand{\Comp}{\rul{Compute}}
%\newcommand{\Vacate}{\rul{Vacate}}
\newcommand{\Move}{\rul{Move}}
\newcommand{\ruSwing}{\rul{Swing}}
\newcommand{\Transfer}{\rul{Transfer}}
\newcommand{\UsefulComp}{\rul{UsefulComp}}
\newcommand{\WriteRulesBit}{\rul{WriteRulesBit}}
\newcommand{\Zigzag}{\rul{Zigzag}}


\begin{document}

\title{A reliable Turing machine}

\author{Ilir \c{C}apuni \and Peter G\'acs}
% \\ Boston University
% \\ gacs@bu.edu

\maketitle
\thispagestyle{empty}

\begin{abstract}
The title says it.
\end{abstract}

\section{Introduction}

\subsection{To be written}

\subsection{Turing machines}\label{sec:TM}

Our contribution uses one of the standard definitions of a Turing
machine.

    A Turing machine $M$ is defined by a tuple
        \begin{align*}
             (\Gamma, \Sigma,\delta, q_{\start},F).
        \end{align*}
    Here, $\Gamma$ is a finite set of \df{states},
    $\Sigma$ is a finite alphabet, and
        \begin{align*}
             \delta\colon\Sigma\times \Gamma\to \Sigma\times\Gamma\times\{-1,0,+1\}
        \end{align*}
    is the transition function.
The tape alphabet $\Sigma$ contains at least the distinguished
symbols $\blank,0,1$ where $\blank$ is called the \df{blank symbol}.
The state $q_{\start}$ is called the \df{starting state}, and
there is a set $F$ of \df{final states}.

    The tape is blank at all but finitely many positions.

    A \df{configuration} is a tuple
        \begin{align*}
             (q,A,\h),
        \end{align*}
    where $q\in\Gamma$ is the \df{current state}, 
$\h\in\bbZ$ is the current \df{head position}, or \df{observed cell},
and $A\in\Sigma^{\bbZ}$ is the \df{tape content}: 
at position \( p \), the tape contains the symbol \( A[p] \).
If $C=(q,A,\h)$ is a configuration then we will write
        \begin{align*}
             C.\state=q,\quad C.\pos=\h, \quad C.\tape=A.
        \end{align*}
    Here, $A$ is also called the \df{tape configuration}.
    The cell at position $\h$ is the  \df{current cell}.
Though the tape alphabet may contain
non-binary symbols, we will restrict input and output to binary.

    The transition function $\delta$ tells us how to compute the next
    configuration from the present one.
    When the machine is in a state $q$, at tape position $\h$, and
    observes tape cell with content $a$, then denoting
         \begin{align*}
           (a',q',j)=\delta(a,q),
         \end{align*}
    it will change the state to $q'$, change the
    tape content at position \( \h \) to $a'$, and move to tape position to $\h+j$.
    For $q\in F$ we have $a'=a$, $q'\in F$.


\begin{definition}[Fault]\label{def:fault}
    We say that a \df{fault} occurs at time $t$ if the output $(a',q',j)$ of the
    transition function at this time is replaced with some other value
    (which is then used to compute the next configuration).
\end{definition}


\subsection{Codes, and the result}

For fault-tolerant computation, some redundant coding of the information is needed.

\begin{definition}[Codes]\label{def:codes}
    Let $\Sigma_{1},\Sigma_{2}$ be two finite alphabets.
    A \df{block code} is given by a positive integer $Q$---called
    the \df{block size}---and a pair of functions
    \begin{align*}
            \psi_{*} :\Sigma_{2}\to\Sigma_{1}^{Q},
            \quad
            \psi^{*}:\Sigma_{1}^{Q}\to\Sigma_{2}
    \end{align*}
    with the property $\psi^{*}(\psi_{*}(x))=x$.
It is extended to strings by encoding each letter individually:
\( \psi_{*}(x_{1},\dots,x_{n})=\psi_{*}(x_{1})\dotsm\psi_{*}(x_{n}) \).
\end{definition}

For ease of spelling out a result, let us consider only computations whose outcome
is a single symbol, at tape position 0.

\begin{theorem}\label{thm:main-main}
There is a Turing machine $M_{1}$ with a 
function $a\mapsto a.\Output$ defined on its alphabet, 
such that
for any Turing machine $G$ with alphabet $\Sigma$ and state set $\Gamma$
there are $0\le\varepsilon <1$ and \( \alpha_{1},\alpha_{2}>0 \) 
with the following property.

For each input length \( n=|x| \) a block code
$(\phi_*, \phi^*)$ of blocksize \( Q=O((\log n)^{\alpha_{1}}) \) can be constructed 
such that the following holds.

Let $M_1$ start its work from its initial state,
and the initial tape configuration \( \xi=(q_{\start},\phi_{*}(x),0) \).
Assume further that
during its operation, faults occur independently at random
with probabilities $\le \varepsilon$.

Suppose that on input $x$ machine $G$ reaches a final state at time $t$ and writes
value \( y \) at position 0 of the tape.
Then denoting by \( \eta(u) \) the configuration machine \( M_{1} \) at time \( u \),
at any time \( t' \) after
 \begin{align*}
   t\cdot (\log t)^{\alpha_{2}},
 \end{align*}
we have \( \eta(t').\tape[0].\Output= y \)
with probability at least $1 - O(\varepsilon)$.
\end{theorem}

We emphasize that the actual
code \( \phi \) of the construction will depend on \( n \) only in a simple way:
it will be the ``concatenation'' of one and the same fixed-size
code with itself, \( O(\log\log n) \) times.


\section{Overview of the construction}

A Turing machine that simulates ``reliably'' any other
Turing machine even when it is subjected to isolated bursts of faults of constant size,
is given in~\cite{burstyTuring12Conf}.
By \df{reliably} we mean that the 
simulated computation can be decoded from the history
of the simulating machine despite occasional damages.


\subsection{Isolated bursts of faults}\label{sec:bursts}

Let us give a brief overview of a machine $M_1$ that
can withstand isolated bursts of faults, as most of its construction will be reused
in the probabilistic setting.

Let us break up the task of error correction into several 
problems to be solved.
The solution of one problem gives rise to another problem one, 
but the process converges.
\begin{flushdescription}
\item[Redundant information] The tape information of the simulated Turing machine
will be stored in a redundant form, more precisely in the form of a block code.
\item[Redundant processing] The block code will be decoded, the retrieved information 
will be processed, and the result recorded.
To carry out all this in a way that limits the propagation of faults, the tape will be split
into tracks that can be handled separately, and the major processing steps will be 
carried out three times within one work period.
\item[Local repair] All the above process must be able to recover from a local burst of faults.
For this, it is organized into a rigid, locally checkable structure
with the help of local addresses, and some other tools like sweeps and 
zigging.
The major tool of local correction, the local recovery procedure, turns out to be the most
complex part of the construction.
\item[Disturbed local repair] A careful organization of the recovery procedure
makes sure that even if a new burst interrupts it (or jumps into its middle),
soon one or two new invocations of it will finish the job (whenever needed).
\end{flushdescription}

Here is some more detail.

Each tape cell of the simulated machine \( M_{2} \) will be represented by a block of
size \( Q \) of the simulating machine \( M_{1} \) called a \df{colony}.
Each step of \( M_{2} \) will be simulated by a computation of \( M_{1} \) called
a \df{work period}.
During this time, the head of \( M_{1} \) makes a number of sweeps over the
current colony, decodes the represented cell symbol and state,
computes the new state, and transfers the necessary information to the 
neighbor colony that corresponds to the new position of the head of \( M_{2} \).

In order to protect information from the propagation of errors,
the tape of \( M_{1} \) is subdivided into \df{tracks}: each track corresponds to a 
\df{field} of a cell symbol of \( M_{1} \) viewed as a data record.
Each stage of computation will be repeated three times.
The results will be stored in separate tracks, and a final cell-by-cell majority vote
will recover the result of the work period from them.

All this organization is controlled by a few key fields, for example a field
called \( \cAddr \) showing the position of each cell in the colony, and a field
\( \cSweep \) showing the last sweep of the computation (along with its direction)
that has been performed already.
The technically most challenging part of the construction is the protection of this
control information from bursts.

For example, a burst can reverse the head in the middle of a sweep.
Our goal is that such structural disruptions be discovered locally, so
we cannot allow the head to go far from the place where it was turned back.
Therefore the head's movement will not be straight even during a single
sweep: it will make frequent short switchbacks (zigzags).
This will trigger alarm, and the start of a recovery procedure if for example
a turn-back is detected.

It is a significant challenge that the recovery procedure
itself can be interrupted (or started) by a burst.


\subsection{Hierarchical construction}

In order to build a machine that can resist faults 
occuring independently of each other with some small probability,
we take the approach suggested in~\cite{Kurd78},
and implemented in~\cite{Gacs1dim86} and~\cite{GacsSorg01}
for the case of one-dimensional cellular automata, with some ideas
from the tiling application of~\cite{DurandRomashShenTiling12}.
We will build a \df{hierarchy of simulations}:
machine $M_1$ simulates machine $M_2$ which 
simulates machine $M_3$, and so on.
For simplicity we assume all these machines have the same program,
and all simulations have the same blocksize.

One cell of machine $M_3$ is simulated by one colony of machine $M_2$.
Correspondingly, one cell of $M_2$ is simulated by
one colony of machine $M_1$.
So one cell of $M_3$ is simulated by $Q^2$ cells of $M_1$.
Further, one step of machine $M_3$ is simulated by one
work period of $M_2$ of, say, \( O(Q^{2}) \) steps.
One step of $M_2$ is simulated by one work period of $M_1$,
so one step of $M_3$ is simulated by $O(Q^{4})$ steps of $M_1$.

Per construction, machine $M_1$ can withstand
bursts of faults whose size is $\le \beta$ for some constant parameter \( \beta \), that
are separated by some $O(Q^{2})$ fault-free steps.
Machines $M_2$, $M_3$, \( \dots \) have the same program, so it
would be natural to expect that machine
$M_1$ can withstand also some \emph{additional}, larger bursts
of size $\le \beta Q$ if those are separated by at least $O(Q^{4})$ steps.

But a new obstacle arises.
On the first level, damage caused by a big burst spans several colonies.
The repair mechanism of machine $M_1$ outlined in Section~\ref{sec:bursts} above 
is too local to recover from such extensive damage.
This cannot be allowed, since then the whole hierarchy would stop working.
So we add a new mechanism to \( M_{1} \) that, more modestly,
will just try to restore a large enough portion of the
tape, so it can go on with the simulation of $M_2$, even if all 
original information was lost.
For this, \( M_{1} \) may need to rewrite an area as large as a few colonies.

This will enable the low-level recovery procedure of 
machine $M_{2}$ to restore eventually a higher-level order.

All machines above $M_1$ in the hierarchy are
``virtual'': the only hardware in the construction is machine $M_1$.

A tricky issue is ``forced self-simulation'': while we are constructing machine $M_1$
we want to give it the feature that it will simulate a machine \( M_{2} \) that
works just like \( M_{1} \).
The ``forced'' feature means that this simulation should
work without any written program (that could be corrupted).

This will be achieved by
a construction similar to the proof of the Kleene's fixed-point 
theorem (also called recursion theorem).
We first fix a (simple) programming language to express the transition
function of a Turing machine.
We write an interpreter for it in this same language (just as compilers for the 
C language are sometimes written in C).
The program of the transition function of \( M_{2} \)
(essentially the same as that of \( M_{1} \))
in this language, is a string that will be
``hard-wired'' into the transition function of \( M_{1} \), 
so that \( M_{1} \), at the start of each work period, can write
it on a working track of the base colony.
Then the work period will interpret it, 
applying it to the data found there, resulting
in the simulation of \( M_{2} \).

In this way, an infinite sequence of simulations arises, in order
to withstand larger and larger but sparser and sparser bursts of faults.

Since the $M_1$ uses the universal interpreter, which in turns
simulates the same program, it is natural to ask
how  machine $M_1$ simulates a given Turing machine $G$ that does the 
actual useful computation?
For this task, we set aside a separate track 
on each machine $M_i$, on which some arbitrary other Turing machine can be
simulated.
The higher the level of the machine \( M_{k} \) that performs this
``side-simulation'', the higher the reliability.
Thus, only the simulations \( M_{k}\to M_{k+1} \) are forced, without program
(that is a hard-wired program):
the side simulations can rely on written programs, since the firm
structure in the hierarchy \( M_{1},M_{2},\dots \) will support them reliably.



\subsection{From combinatorial to probabilistic noise}

The construction we gave in the previous subsection
was related to increasing bursts that are not frequent.
In essence, that noise model is combinatorial.
To deal with probabilistic noise combinatorially,
we stratify the set of faulty times $\Noise$ as follows.
For a series of parameters \( \beta_{k}, V_{k} \),
we first remove ``isolated bursts'' of type \( \pair{\beta_{1}}{V_{1}} \) of elements of this set.
(The notion of ``isolated bursts'' of type \( \pair{\beta}{V} \)
will be defined appropriately.)
Then, we remove isolated bursts of type \( \pair{\beta_{2}}{V_{2}} \) from the remaining set,
and so on.
It will be shown that with the appropriate choice of parameters, with probability 1,
eventually nothing is left over from the set $\Noise$.

A composition of two reliable simulations is even more reliable.
We will see that a sufficiently large hierarchy of such
simulations resists probabilistic noise.


\subsection{Difficulties}\label{sec:novelties}

Let us spell-out some of the main problems that the paper deals with, 
and some general ways they will be solved or avoided.


\begin{bullets}

\item A large burst of $M_1$ can modify the order of
entire colonies or create new ones with gaps between them.

To overcome this problem conceptually, we 
introduce the notion of a \df{generalized Turing machine}
allowing for non-adjacent cells.
Each such machine has a parameter $\B$ called the \df{cell body size}.
The cell body size of a Turing machine in Section~\ref{sec:TM} would still remain
$1$.

    \item What to do when the head is in a middle of an empty area
       where no structure exists?
To ensure reliable passage across such areas,
we will try to keep everything filled with cells, even if these are
not part of the main computation.

    \item Noise can create areas over
     which the predictability of the simulated machine is limited.
     In these islands the (on this level) invisible structure
     of the underlying simulation may be destroyed.
     These areas should not simply be considered blank, since
     blankness implies predictable behavior.
     They will be called \df{damaged}.
     When the head is in or near damage then even in the absence of new faults,
     the predictability of the behavior of the (simulated) machine will be
     severely limited.

\item Due to limited predictability over damage, the definition of the generalized
Turing machine stipulates a certain ``magical'' erasure of damage in case the
head stays long enough on it.
(Of course, this property needs to be implemented in simulation, which is one of the
main burdens of the actual construction.)
Once damage is erased the area will be re-populated with new cells.
Their content is not important, what matters is the restoration of predictability.

        \item If local repair fails, a special rule will be invoked that reorganizes a
larger part of the tape (of the size of a few colonies instead of only a few cells).
This is the mechanism enabling the ``magical'' restoration on the next level.

\end{bullets}

\subsection{A shortcut solution}

A fault-tolerant one-dimensional cellular automaton is constructed
in~\cite{GacsSorg01}.
If our Turing machine could just simulate such an automaton, it would become
fault-tolerant.
This can indeed almost be done provided that the size of the computation is known in advance.
The cellular automaton can be made finite, and we could define
a ``kind of'' Turing machine with a \emph{circular tape} simulating it.
But this solution requires input size-dependent hardware.

It seems difficult to define a fault-tolerant sweeping 
behavior on a regular Turing machine needed to 
simulate cellular automaton, without recreating
an entire hierarchical construction---as we are doing here.


\section{Notational conventions}\label{sec:notational-conventions}

Most notational conventions given here are common; some other ones will
also be useful.

\begin{flushdescription}

\item [Natural numbers and integers] 
By $\bbZ$ we denote the set of integers.
\begin{align*}
   \bbZ_{>0}&=\setOf{x}{x\in \bbZ,\;  x>0}, \\
   \bbZ_{\ge 0}&=\bbN=\setOf{x}{x\in \bbZ,\;  x\ge 0}.
\end{align*}

\item [Intervals]
We use the standard notation for intervals:
\begin{align*}
   \clint{a}{b}&=\setOf{x}{a\le x \le b},\quad \lint{a}{b}=\setOf{x}{a\le x < b}, \\
   \rint{a}{b}&=\setOf{x}{a< x \le b}, \quad  \opint{a}{b}=\setOf{x}{a< x < b}.
\end{align*}
We will also write \( \lint{a}{b} \) in place of \( \lint{a}{b}\cap \bbZ \), 
whenever this leads to no confusion.
Instead of \( \lint{x+a}{x+b} \), sometimes we will write 
\begin{align*}x + \lint{a}{b}.\end{align*}

\item [Ordered pairs]
Ordered pairs are also denoted by \( \pair{a}{b} \),
but it will be clear from the context whether we are
referring to an ordered pair or open interval.

\item [Comparing the order of a number and an interval]
For a given number $x$ and interval $I$, we
write
\begin{align*} x \ge I \end{align*}
if for every $y\in I$,  $x \ge y$.

\item [Distance]
The distance between two real numbers $x$ and $y$ is defined
in a usual way:
\begin{align*}
    d(x,y)= |x-y|.
\end{align*}

The \df{distance of a point $x$ from interval $I$}  is
\begin{align*}
    d(x,I)= \min_{y\in I}d(x,y).
\end{align*}

\item [Ball, neighborhood, ring, stripe]
A \df{ball of radius $r>0$, centered at $x$} is
\begin{align*}
    B(x,r)= \setOf{y}{d(x,y)\le r}.
\end{align*}
An \df{$r$-neighborhood of interval } $I$ is
\begin{align*}
    \setOf{x}{d(x,I)\le r}.
\end{align*}
An \df{$r$-ring} around interval $I$ is
\begin{align*}
    \setOf{x}{d(x,I)\le r \txt{ and } x \notin I}.
\end{align*}
A \df{$r$-stripe to the right of interval $I$} is
\begin{align*}
    \setOf{x}{d(x,I)\le r \txt{ and } x \notin I \txt{ and } x>I}.
\end{align*}

\item[Logarithms] Unless specified differently,
the base of logarithms throughout this work is 2.

\end{flushdescription}


\section{The model}

Standard Turing machines do not have
operations like ``creation'' or ``killing'' of cells, nor
do they allow for cells to be non-adjacent.
We introduce here a \df{generalized Turing machine}.
It depends on an integer $\B \ge 1$ that denotes the cell body size,
and an upper bound $\Tu$ on the transition time.
These parameters are convenient since they provide the illusion that the different Turing
machines in the hierarchy of simulations all operate on the same linear space.
Even if the notions of cells, alphabet
and state are different for each machine of the hierarchy, 
at least the notion of a \emph{location
on the tape} is the same.


\begin{definition}[Generalized Turing machine]\label{def:gen-TM}
    A \df{generalized Turing machine} $M$ is defined by a tuple
        \begin{align}\label{eq:gen-TM}
             (\Gamma, \Sigma, \delta, \G, \cDir, q_{\start},F, \B, \Tu),
       \end{align}
    where $\Gamma$ and $\Sigma$ are finite sets
    called the \df{set of states} and the \df{alphabet} respectively,
        \begin{align*}
             \delta: \Sigma\times \Gamma
             \to \Sigma\times \Gamma\times\{-1,0,1\},
        \end{align*}
    is the \df{transition function}
The function (``field'') \( \G\colon\Gamma\to\{\true,\false\} \) of the state will show whether
the last move was from a non-adjacent cell.
The function (``field'') \( \cDir\colon\Sigma\to\{-1,0,1\} \) of the cell content
needs to always point towards the head, so 
the transition function \( \delta \) is required to have the property that
if \( (a',q',j)=\delta(a,q) \) then \( a'.\cDir=j \).

The role of starting state $q_{\start}$ and final states in \( F \) is as before.
The integer $\B\ge 1$ is called the \df{cell body size},
and the real number $\Tu$ is a bound on the transition time.

Among the elements of the tape alphabet $\Sigma$, 
we distinguish the elements $0,1,\Bad,\Vacant$.
The role of the symbols $\Bad$ and \( \Vacant \) will be clarified below.
\end{definition}


\begin{definition}[Configuration]\label{def:config}
     Consider a generalized Turing machine~\eqref{eq:gen-TM}.
    A \df{configuration} is a tuple
        \begin{align*}
             (q,A,\h,\hc),
        \end{align*}
    where $q\in\Gamma$, $\h,\hc\in\bbZ$ and $A\in\Sigma^{\bbZ}$.
As before, the array \( A \) is the tape configuration.
The \df{damage set} is defined as
     \begin{align*}
          A.\Damage=\setOf{p}{A[p]=\Bad}.
     \end{align*}
We say that there is a \df{cell} at position $p\in\bbZ$ if
$A[p]\notin \set{\Vacant, \Bad}$.
In this case, we call the interval $p+\lint{0}{\B}$ the \df{body} of this cell.
Cells must be at distance \( \ge\B \) from each other, that is their
bodies must not intersect.
They are called \df{adjacent} if the distance is exactly \( \B \).

For all cells \( p \), the value $A[p].\cDir$ is required to point towards 
the head \( \h \), that is 
 \begin{align*}
   A[p].\cDir=\sign(\h-p).
 \end{align*}
Whenever $A[\h]\neq \Bad$ there must be a
cell at position \( \hc \) called the \df{current cell} with a body 
within $2.5\B$ from $\h$.

The array \( A \) is $\Vacant$ everywhere but in finitely many positions.

Let
    \begin{align*}
         \Configs_{M}
    \end{align*}
    denote the set of all possible configurations
    of a Turing machine $M$.
\end{definition}

\section{Trajectory, simulation, and the tower}

It is natural to name a sequence of configurations that is conceivable as a computation
(faulty or not) of a Turing machine as ``history''.
The histories that obey the transition function then could be called ``trajectories''.
In what follows we will 
stretch and generalize this notion to encompass also some limited violations of the
transition function.

In connection with any underlying Turing machine with a given starting configuration, we will
denote by
\begin{align}\label{eq:noise-first}
   \Noise\subseteq \bbZ\times \bbZ_{\ge 0}
\end{align}
the set of space-time points $\pair{p}{t}$, such that
a fault occurs at time $t$ when the head is at position $p$.

\begin{definition}[History]\label{def:history}
  \begin{sloppypar}
    Let us be given a generalized Turing machine~\eqref{eq:gen-TM}.
    Consider a sequence $\eta = (\eta(0), \eta(1), \dots)$ of configurations with
    \( \eta(t) = \) \( (q(t), A(t), \h(t), \hc(t)) \), along with a noise set $\Noise$.
    The \df{switching times} of this sequence 
are the times when one of the following can change:
the state, the position \( \hc(t) \) of the current cell, or the state of the current cell.
The interval between two consecutive switching times is the \df{dwell period}.
The pair
      \end{sloppypar}
    \begin{align*}
       (\eta,\Noise)
    \end{align*}
    will be called a \df{history} of machine $M$ if the following conditions hold.
        \begin{bullets}
            \item We have $|\h(t) - \h(t')| \le |t' - t|$.

            \item In two consecutive configurations, the content \( A(t)[p] \) of the positions not in
                  $\h(t) + \lint{-\B}{\B}$, remains the same:
                  $A(t+1)[n] = A(t)[n]$ for all $n \notin \h(t) + \lint{-\B}{\B}$.

            \item At each noise-free switching time the head is on the new current cell, that is
\( \hc(t)=\h(t) \).
In particular, when at a switching time a current cell becomes
$\Vacant$, the head must already be on another (current) cell.

            \item The length of any dwell period in which the head does not intersect
noise or damage, is at most \( \Tu \).

        \end{bullets}
    Let
        \begin{align*}
            \Histories_{M}
        \end{align*}
    denote the set of all possible histories of $M$.

We say that a cell \df{dies} in a history it becomes $\Vacant$.

\end{definition}

The transition function $\delta$ of a generalized Turing machine
imposes constraints on histories: those
histories obeying the constraints will be called trajectories.

\begin{definition}
Suppose that the machine is in a state $q$, with current cell $x$, 
with cell content $a$, let $(a',q',j) = \delta(a,q)$.
We say that the new content of cell \( x \) (including when it dies),
the direction of the new position \( y \) from \( x \),
and the new state are \df{directed by the transition function} if
the following holds.
The new content of \( x \) is \( a' \), and
the direction of \( y \) from \( x \) is \( j \).
In the new state \( q \) we have \( q.\G=\false \)
if \( j=0 \) or the new current cell is adjacent, and \( \true \) otherwise.  
\end{definition}

\begin{definition}[Trajectory]\label{def:traj}
The definition of a trajectory depends on a constant 
\begin{align*}
   c_{1}>0
 \end{align*}
to be fixed later.

\begin{sloppypar}
   A history  $(\eta, \Noise)$ of a generalized Turing 
machine~\eqref{eq:gen-TM} with \(\eta(t) =\)
\( (q(t), A(t), \h(t), \hc(t)) \)
is called a \df{trajectory} of $M$ if the following conditions hold, during any 
noise-free time interval.
Denote 
\begin{align*}
     (a, q, p) &= (A(t)[\hc(t)], q(t),\hc(t)),
\\ (a',q',j)   &= \delta(a,q).
 \end{align*}
  \end{sloppypar}
\begin{flushdescription}

\item[Transition function]\label{i:def.traj.transition}
Suppose that there is a switch, and the shortest interval
containing the body of the current cell \( x \) and the
new cell \( y \) is at a distance at least \( 0.5\B \)
from damage.

Then the new state, the cell content of \( x \) (including when it dies), and
the direction of \( y \) from \( x \) are directed by the transition function.
If \( y \) did not exist before then it is adjacent to \( x \).

Further the length of the dwell period is bounded by \( \Tu \).

\item[Spill]\label{i:bound-on-damage}
  The damage may spill only to at most a distance \( 2\B \) beyond its 
place at time \( t \).

\item[Attacking damage from outside] \label{i:def.traj.attack-from-outside}
Suppose that the body of the current cell $x$ is disjoint from damage, 
and at least $0.5\B$ removed from damage on the left.
Suppose further that the transition function directs the head towards the right.
Then whenever the head comes left  of \( x \) again, the  
damage will recede by a distance at least \( \B \) from the body of \( x \).

A similar property is required when ``left'' and ``right'' are interchanged.

\item[Clearing damage from within] \label{i:def.traj.damage-within}
Let $ I $ be a space interval of size $ 3\B $.
If the head spends a total time of at least $ c_{1}\Tu $ inside $ I $ (while possibly
entering repeatedly), then $ I $ becomes damage-free.

\end{flushdescription}
\end{definition}


Until this moment, we used the term ``simulation'' to denote
a correspondence between configurations of
two machines which remains preserved during the computation.


\begin{definition}[Simulation] \label{def:simulation-central}
Let $M_{1},M_{2}$ be two generalized Turing machines, and let
\begin{align*}
    \varphi_{*}:\Configs_{M_{2}} \to \Configs_{M_{1}}
\end{align*}
be a mapping from configurations of $M_{2}$
to those of $M_{1}$, such that it maps
starting configurations into starting configurations.
We will call such a map a \df{configuration encoding}.

Let
\begin{align*}
   \Phi^{*}:\Histories_{M_{1}} \to \Histories_{M_{2}}
\end{align*}
be a mapping.
The pair $(\phi_{*}, \Phi^{*}) $
is called a \df{simulation} (of $ M_{2} $ by $ M_{1} $) if for every
trajectory $ (\eta, \Noise) $ with initial
configuration $ \eta(0)=\phi_{*}(\xi) $,
the history $ (\eta^{*},\Noise^{*})=\Phi^{*}(\eta,\Noise) $ is
a trajectory of machine $ M_{2} $.

We say that $M_{1}$ \df{simulates} $M_{2}$ if there is a simulation
$(\varphi_{*},\Phi^{*})$ of $M_{2}$ by $M_{1}$.
\end{definition}


\section{Hierarchical codes}\label{sec:hier-codes}

Recall the notion of a code in Definition~\ref{def:codes}.

\begin{definition}[Code on configurations]\label{def:configuration-code}
Consider two generalized Turing macines \( M_{1},M_{2} \) with the corresponding
state spaces, alphabets and transition functions, and an integer \( Q\ge 1 \).
We require
\begin{align}\label{eq:B_2-B_1-Q}
  \B_{2} = Q \B_{1}.
\end{align}
Assume that a block code
\begin{align*}
   \psi_{*}:\Sigma_{2}\times(\Gamma_{2}\cup\{\emptyset\})\to\Sigma_{1}^{Q}
 \end{align*}
is given, with an appropriate decoding function, $\psi^{*}$.
With \( \pair{a}{q}\in\Sigma_{2}\times(\Gamma_{2}\cup\{\emptyset\}) \),
symbol \( a \) is interpreted the content of some tape square.
The value \( q \) is the state of \( M_{2} \) provided the head is observing this square,
and \( \emptyset \) if it is not.

Block code $(\psi_{*},\psi^{*})$ gives rise to a 
\df{code on configurations}, that is a pair of functions
    \begin{align*}
        \phi_{*} :\Configs_{M_2} \to \Configs_{M_1},
        \quad
        \phi^{*}:\Configs_{M_1} \to \Configs_{M_2}
    \end{align*}
    that encodes configurations of $M_2$ into configurations of $M_{1}$.

Let $\xi$ be a configuration of $M_2$.
We set $\phi_*(\xi).\pos = \xi.\pos$, \(\phi_{*}.\state=  \) 
the starting state of \( M_{1} \),
\begin{align*}
 \phi_*(\xi).\tape[i\B_2, \dots, (i+1)\B_2 - \B_1] = \psi_*(\xi.\tape[i], s)
 \end{align*}
where \( s=\xi.\state \) if $i = \xi.\pos$, and $\emptyset$ otherwise, with
a slight \emph{modification}: 
$\phi_*(\xi).\tape[i].\cDir$, is set to point towards the head \( \xi.\pos \).
Decoding is the inverse of this process (need not succeed on all possible configurations
of \( M_{1} \)).
 \end{definition}

 Not all configurations can be obtained by encoding.
 We distinguish those that can.

 \begin{definition}[Code configuration]\label{def:code-config}
   A configuration $\xi$ is called a \df{code configuration} if 
it has the form \( \xi=\phi_{*}(\zeta) \).
 \end{definition}


\begin{definition}[Hierarchical code]\label{def:hierarchical-code}
For $k\ge 1$, let $\Sigma_k$ be an alphabet, $\Gamma_k$ be
a set of states of a generalized Turing machine $M_k$.
Let $Q_k>0$ be an integer colony size, let $\phi_k$
be a code on configurations defined by a block code
  \begin{align*}
       \psi_k: \Sigma_{k+1}\times(\Gamma_{k+1}\cup\{\emptyset\})
       \rightarrow \Sigma_k^{Q_k}
  \end{align*}
as in Definition~\ref{def:configuration-code}.
The sequence of triples $(\Sigma_k,\Gamma_k,  \phi_k)$, ($k\ge 1)$,  is
called a \df{hierarchical code}.
For the given hierarchical code, the configuration \( \xi^{1} \) of \( M_{1} \)
is called a \df{hierarchical code configuration} if a sequence
of configurations $\xi^{2},\xi^{3},\dots$ of \( M_{2},M_{3},\dots \) exists with
\begin{align*}
 \xi^{k}=\phi_{*k}(\xi^{k+1})
 \end{align*} 
for all \( k \).
(Of course, then whole sequence is determined by \( \xi^{1} \).)
\end{definition}

Let us give a name to the object that we want to construct.

\begin{definition}[A tower]\label{def:tower}
Let $M_1$, $M_2$, \dots, be a sequence of generalized Turing machines,
let \( \phi_{1} \), \( \phi_{2} \) , \(\dots \) be a hierarchical code for this sequence,
let \( \xi^{1} \) be a hierarchical code configuration for it, where \( \xi^{k} \) is an
initial configuration of \( M_{k} \) for each \( k \).
Let further be a sequence of mappings \( \Phi_{1} \), \( \Phi_{2} \), \( \dots \) be
given such that for each \( k \), the pair $(\phi_{k*},\Phi_{k}^{*})$,
is a simulation of \( M_{k+1} \) by \( M_{k} \).
Such an object is called a \df{tower}.
\end{definition}

The main task of the work will be the definition of a tower, since the simulation
property is highly nontrivial.


\section{Layering the noise}\label{sec:noise}

Let us introduce a technique connecting the combinatorial and probabilistic
noise models.

\begin{definition}[Centered rectangles, isolation]
Let $\vek{r}=\pair{r_1}{r_2}$, $r_1, r_2\ge 0$,
be a two-dimensional nonnegative vector.
An \df{rectangle} of radius $\vek{r}$ \df{centered} at $\vek{x}$ is
\begin{align}\label{eq:ball1}
  B(\vek{x},\vek{r}) = \setOf{\vek{y}}{|y_i - x_i| \le r_i, i=1,2}.
\end{align}  
Let \( E\subseteq \bbZ^{2} \) be a two-dimensional set.
A point $\vek{x}$ of \( E \) is \df{$\pair{\vek{r}}{\vek{r}^*}$-isolated} if
\begin{align*}
  E \cap B(\vek{x},\vek{r}^*)\subseteq B(\vek{x}, \vek{r}).
 \end{align*}
  Let
\begin{align}
  D(E,\vek{r}, \vek{r}^*) =
     \setOf{\vek{x}\in E}{\vek{x} \txt{ is not } (\vek{r}, \vek{r}^*)\txt{-isolated
  from } E}.
\end{align}
\end{definition}

\begin{definition}[Sparsity]
Let \( \beta\ge 9 \) be a constant, and let 
\( 0<\B_{1}<\B_{2}<\dotsm \), \( \Tu_{1}<\Tu_{2}<\dotsm \) be 
sequences of positive integers to be fixed later.

For a two-dimensional set \( E \), let $E^{(1)} = E$.
For $k>1$ we define recursively:
\begin{align}\label{eq:noise^k}
    E^{(k+1)} = D(E^{(k)}, \beta\pair{\B_k}{\Tu_{k}}, \pair{\B_{k+1}}{\Tu_{k+1}}).
\end{align}
Set $E^{(k)}$ is called the \df{$k$-th residue} of $E$.

Set $E$ is \df{$(\vek{r}, \vek{r}^*)$-sparse} 
if $D(E, \vek{r}, \vek{r}^*)=\emptyset$, that is 
it consists of $(\vek{r}, \vek{r^*})$-isolated points.
It is \( k \)-\df{sparse} if $E^{(k+1)}=\emptyset$.
It is simply \df{sparse} if \( \bigcap_{k}E^{(k)}=\emptyset \).
\end{definition}

The following lemma connects the above defined sparsity notions to the requirement
of small fault probability.

\begin{lemma}[Sparsity]\label{lem:sparsiness}
Let $1=\B_{1}\le  \B_{2}\le\dotsm$ and 
$1=\Tu_{1}<T_{2}<\dotsm$ be sequences of integers with
$Q_k = B_{k+1}/B_{k}$, $\U_k = \Tu_{k+1}/\Tu_{k}$, and
\begin{align}\label{eq:assumption}
  \lim_{k\rightarrow\infty}\frac{\log(\U_{k} Q_k)}{1.5^k}=0,
\end{align}
For sufficiently small \( \varepsilon \), for every \( k\ge 1 \) the following holds.
Let \( E\subseteq \bbZ\times \bbZ_{\ge 0} \)
be a random set with the property that each pair $\pair{p}{t}$ belongs to \( E \)
independently from the other ones with probability $\le \varepsilon$.

Then for each point $\vek{x}$  and each \( k \),
 \begin{align*}
   \Pbof{B(\vek{x},(\B_k, \Tu_{k}))\cap E^{(k)}\neq\emptyset} <2\varepsilon \cdot 2^{-1.5^{k}}.
 \end{align*}
As a consequence, the set \( E \) is sparse with probability 1.
\end{lemma}

\begin{proof}
Let \( k=1 \).
Rectangle \( B(\vek{x},(\B_1, \Tu_{1})) \) is a single point, hence
the probability of our event is $<\varepsilon$.
Let us prove the inequality by induction, for \( k+1 \).

Note that our 
event depends at most on the rectangle \( B(\vek{x},3(\B_{k}, \Tu_{k})) \).
Let
\begin{align*}
   p_{k}=2\varepsilon\cdot 2^{-1.5^{k}}.
\end{align*}
Suppose \( \vek{y} \in E^{(k)}\cap B(\vek{x},(\B_{k+1}, \Tu_{k+1})  ) \).
Then, according to the definition of $E^{(k)}$,  there is a point
\begin{align}\label{eq:sparse-as}
 \vek{z} \in
 B(\vek{y},\Tu_{k+1})\cap E^{(k)}\setminus B(\vek{y},\beta(\B_{k}, \Tu_{k})).
 \end{align}
Consider a standard partition of the (two-dimensional) space-time into
rectangles \( K_{p}=\vek{c}_{p}+\lint{-\B_{k}}{\B_k}\times \lint{-\Tu_{k}}{\Tu_{k}} \)
with centers \( \vek{c}_{1},\vek{c}_{2},\dots \).
The rectangles \( K_{i},K_{j} \) containing \( \vek{y} \) and \( \vek{z} \)
respectively intersect \( B(\vek{x}, 2(\B_{k+1}, \Tu_{k+1}) \).
The triple-size rectangles 
\( K'_{i}=c_{i} + \lint{-3\B_{k}}{3\B_k}\times \lint{-3\Tu_{k}}{3\Tu_{k}} \) and
\( K'_{j} \) are disjoint, since \eqref{eq:sparse-as} implies
 \( |y_1 - z_1|>\beta\B_{k} \) and \( |y_2 - z_2|>\beta\Tu_{k} \).

The set \( E^{(k)} \) must intersect two rectangles \( K_{i} \),
\( K_{j} \) of size \( 2(\B_{k}, \Tu_{k}) \) separated by at least \( 4(\B_{k}, \Tu_{k}) \),
of the big rectangle \( B(\vek{x},2(\B_{k+1}, \Tu_{k+1})) \).

By the inductive hypothesis, the event \( \cF_{i} \) that
\( K_{i} \) intersects \( E_{k} \) has probability bound \( p_{k} \).
It is independent of the event \( \cF_{j} \), since these events depend
only on the triple size disjoint rectangles \( K'_{i} \) and \( K'_{j} \).

The probability that both of these events hold is at most \( p_{k}^{2} \).
The number of possible rectangles
\( K_{p} \) intersecting \( B(\vek{x},2(\B_{k+1}, \Tu_{k+1})) \) is
at most
\( C_{k}:=((2\U_{k} Q_{k})+2)^{2} \), so the number of possible pairs of rectangles
is at most \( C_{k}^{2}/2 \), bounding the probability of our event by
 \begin{align*}
   C_{k}^{2}p_{k}^{2}/2
    &=
      2 C_{k}^{2}\varepsilon^{2} 2^{-1.5^{k+1}}\cdot 2^{-0.5\cdot 1.5^{k}}
   \\ &=2\varepsilon 2^{-1.5^{k+1}} \cdot \varepsilon
        C_{k}^{2}2^{-0.5\cdot 1.5^{k}}.
 \end{align*}
Since $\lim_{k}\frac{\log{(\U_{k} Q_k)}}{1.5^k}=0$,
the last factor is \( \le 1 \) for sufficiently small  \( \varepsilon \).
\end{proof}


\section{Universal Turing machine}\label{sec:UTM}

We will describe our construction in terms of
universal Turing machines,
operating on binary strings as inputs and outputs.
We define universal Turing machines in a way that allows
for rather general ``programs''.

 \begin{definition}[Standard pairing]
For a (possibly empty) binary string \( x=x(1)\dotsm x(n) \) let us introduce the map
 \[
   \ang{x} = 0^{|x|}1 x,
 \]
Now we encode pairs, triples, and so on, of binary strings as follows:
 \begin{align*}
        \ang{s,t} &=\ang{s}t,
\\ \ang{s,t,u} &= \ang{\ang{s,t},u},
 \end{align*}
and so on.

From now on, we will assume that our alphabets \( \Sigma \), \( \Gamma \)
are of the form \( \Sigma=\{0,1\}^{s} \), \( \Gamma=\{0,1\}^{g} \), that is 
our tape symbols and machine states are viewed as binary strings of a certain length.
Also, if we write \( \ang{i,u} \) where \( i \) is some number, it is understood
that the number \( i \) is represented in a standard way by a binary string.
\end{definition}

\begin{definition}[Computation result, universal machine]
 Assume that a Turing machine \( M \) starting on binary \( x \),
 at some time \( t \)
 arrives at the first time at some final state.
 Then we look at the longest (possibly empty)
 binary string to be found starting at position
 0 on the tape, and call it the \df{computation result} \( M(x) \).
We will write
 \begin{align*}
   M(x,y)=M(\ang{x,y}),\quad M(x,y,z)=M(\ang{x,y,z}),
 \end{align*}
and so on.

A Turing machine \( U \) is called \df{universal} 
among Turing machines with
binary inputs and outputs, if for every Turing machine \( M \),
there is a binary string \( p_{M} \) such that for all \( x \) we have
\( U(p_{M},x)=M(x) \).
(This equality also means that the computation denoted on the left-hand side 
reaches a final state if and only if the computation on the right-hand side does.)
\end{definition}

Let us introduce a special kind of universal Turing machines, to be
used in expressing the transition functions of other Turing machines.
These are just the Turing machines for which the so-called \( s_{mn} \) theorem
of recursion theory holds with \( s(x,y)=\ang{x,y} \).

\begin{definition}[Flexible universal Turing machine]\label{def:univ-TM}
A universal Turing machine will be called \df{flexible} if 
whenever \( p \) has the form \( p=\ang{p',p''} \) then
\begin{align*}
 U(p,x)= U(p',\ang{p'',x}).
 \end{align*}
Even if \( x \) has the form \(x =\ang{x',x''} \), this definition chooses
\( U(p',\ang{p'',x}) \) over \( U(\ang{p,x'},x'') \), that is starts with 
parsing the first argument
(this process converges, since \( x \) is  shorter than \( \ang{x,y} \)).
\end{definition}

It is easy to see that there are flexible universal Turing machines.
On input \( \ang{p,x} \),
a flexible machine first checks whether its ``program'' \( p \) 
has the form \( p=\ang{p',p''} \).
If yes, then it applies \( p' \) to the pair \( \ang{p'',x} \).
(Otherwise it just applies \( p \) to \( x \).)

\begin{definition}[Transition program]
  Consider an arbitrary Turing machine \( M \) with state set \( \Gamma \), alphabet
\( \Sigma \), and transition function \( \delta \).
A binary string \( \pi \) will be called a \df{transition program} of \( M \) if
whenever \( \delta(a,q)=(a',q',j) \) we have
 \begin{align*}
 U(\pi,a,q)=\ang{a',q',j}.
 \end{align*}
We will also require that the computation induced by the program makes
\( O(|p|+|a|+|q|) \) left-right turns, over a length tape \( O(|p|+|a|+|q|) \).
\end{definition}

The transition program just provides a way to compute
the (local) transition function of \( M \) by the universal machine,
it does not organize the rest of the simulation.

\begin{remark}
 In the construction of universal Turing machines provided by the textbooks
(though not in the original one given by Turing), the program is generally a string
encoding a table for the transition
function \( \delta \) of the simulated machine \( M \).
Other types of program are imaginable: some simple transition functions can
have much simpler programs.
However, our fixed machine is good enough (similarly to the optimal machine
for Kolmogorov complexity).
If some machine \( U' \) simulates \( M \) via a
very simple program \( q \), then
 \begin{align*}
     M(x)=U'(q,x) = U(p_{U'},\ang{q,x}) = U(\ang{p_{U'},q},x),
 \end{align*}
so \( U \) simulates this computation via the program \( \ang{p_{U'},q} \).
\end{remark}


\section{Rule language}\label{sec:language}

In what follows we will describe the generalized Turing machines \( M_{k} \) for all \( k \).
They are all similar, differing only in the parameter \( k \); the most important activity
of \( M_{k} \) is to simulate \( M_{k+1} \).
The description will be uniform, except for the parameter \( k \).
We will denote therefore \( M_{k} \) simply by \( M \), and \( M_{k+1} \)  by \( M^{*} \).
Similarly we will denote the block size \( Q_{k} \) of the block code of the 
simulation simply by \( Q \).

Instead of writing a huge table describing the transition function \( \delta_{k}=\delta \),,
we present the transition function as a set of \df{rules}.
It will be then possible to write one \emph{interpreter} program that carries
out these rules; that program can be written for some fixed flexible 
universal machine $\Un$.

Each rule consists of some (nested) conditional statements,
similar to the ones seen in an ordinary program:
 ``\textbf{if} \textit{condition} \textbf{then} \textit{instruction}
\textbf{else} \textit{instruction}'', 
where the condition
is testing values of some fields of the state and the observed cell, and
the instruction can either be elementary, or itself a conditional statement. 
The elementary instructions are an \df{assignment} of a value to a field
of the state or cell symbol, or a command to move the head.
Rules can call other rules, but these calls will never form a cycle.
Calling other rules is just a shorthand for nested conditions.

Even though rules are written like procedures of a program,
they describe a single transition.
When several consecutive statements are given, then they
%(almost always)
change different fields of the state or
cell symbol, so they can be executed simultaneously.
% Otherwise and in general, even if a field is updated in
% some previous statement, in all following statements that use
% this field, its old value is considered.

Assignment of value $x$ to a field $y$ of the state or cell symbol will
be denoted by $y \gets x$.
We will also use some conventions introduced by the C language:
namely,
$x\gets x+1$ and $x\gets x-1$ are abbreviated to $\increment{x}$ and
$\decrement{x}$ respectively.

Rules can also have parameters, like $\ruSwing(a,b,u,v)$.
Since each rule is called only a constant number of times in the whole program,
the parametrized rule can be simply seen as a shorthand.

Mostly we will describe
the rules using plain English, but it should always be clear that they
are translatable into such rules.


\section{Fields}\label{sec:fields}

\begin{sloppypar}
For the machine \( M \) we are constructing, each state will 
be a tuple \( q=(q_{1},q_{2},\dots,q_{k}) \),
where the individual elements of the tuple will be called \df{fields}, and will
have symbolic names.
For example, we will have fields \( \Addr \) and \( \Drift \),
and may write \( q_{1} \) as \( q.\Addr \) or just \( \Addr \), 
\( q_{2} \) as \( q.\Drift \) or \( \Drift \), and so on.
\end{sloppypar}

Similarly for tape symbols.
In order to distinguish fields of tape symbols from fields of the state,
we will always start the name of a field of the tape symbols by the letter \( c \).
We have seen already one example of this, the field \( \cDir \) of tape symbols
in the definition of a generalized Turing machine.

In what follows we describe some of the most important fields we will use;
others will be introduced later.

A properly formatted configuration of \( M \) splits the tape into blocks of \( Q \)
consecutive cells called \df{colonies}.
One colony of the tape of the simulating
machine represents one cell of the simulated machine.
The colony that corresponds to the cell that the
simulated machine is scanning is called the \df{base colony}
(a precise definition will be based on the actual history of the work of \( M \)).
Once the drift is known, the union of the base colony with the neighbor colony in
the direction of the drift is called the \df{extended base colony} (this notion will 
need to be defined more carefully later).

The states of machine \( M \) will have a field called 
 \begin{align*}
   \Mode\in\{ \Normal,\Recovering, \Healing \}.
 \end{align*}
It shows whet her the machine is engaged in the regular business of simulation 
or in the repair of its own structure.
The normal mode corresponds to the states
where \( M \) is performing the simulation of \( M^{*} \).
The recovery mode tries to correct some perceived fault.
The \( \Healing \) mode turns on when recovery fails.

Similarly, each cell of the tape of \( M \) consists of several fields.
Some of these have names identical to fields of the state.
In describing the transition rule of \( M \) we will write, for example,
\( q.\Addr \) simply as \( \Addr \), and for the corresponding field of the
observed cell symbol \( a \) we will write \( a.\cInfo \), or just \( \cInfo \).
The array of values of the same field of the cells will be called a \df{track}.
Thus, we will talk about the \( \cHold \) track of the tape, corresponding to the
\( \cHold \) field of cells.

% The basic fields of the state and of cells are listed in Section~\ref{sec:fields-list}
% with some hints of
% their function (this does not replace our later definition of the transition function).
Each field of a cell has also a possible value
\( \emptyset \) whose approximate meaning is ``undefined''.

Some fields and parameters are important enough to introduce them right away.
The 
\begin{align*}
   \cInfo,\cState
 \end{align*}
track of a colony of \( M \)
contain the strings that encode the content of the simulated cell of \( M^{*} \) and
its simulated state respectively.
\begin{align*}
 \cProg
 \end{align*}
track stores the program of \( M^{*} \), in an appropriate form 
to be interpreted by the simulation.
The field 
 \begin{align*}
  \cAddr
 \end{align*}
of the cell shows the position of the cell in its colony.
There is a corresponding \( \Addr \) field of the state.
The
 \begin{align*}
 \Sweep
 \end{align*}
field counts the sweeps that the head makes during the work period.
There is a corresponding \( \cSweep \) field in the cell.
The direction \( d\in\{-1,0,1\} \) in which the 
simulated head moves will be denoted by
 \begin{align*}
   \Drift.
 \end{align*}
There is a corresponding field \( \cDrift \).
The number of the last sweep of the work period will depend on the drift \( d \), 
and will be denoted by 
\begin{align}\label{eq:Last}
   \Last(d).
 \end{align}
Cells will be designated as belonging to a number of possible \df{kinds}, signaled by the
field 
\begin{align*}
     \cKind
 \end{align*}
with values
       \begin{align*}
          \Member, \Neighbor, \Bridge, \Stem, \Vacant.
       \end{align*}
The role of cell kinds will be explained later.

It is useful to group some of the fields especially important for the simulation structure into
a ``super'' field,
    \begin{align}\label{eq:Core}
       \Core=(\Addr, \Sweep, \Drift, \Kind), 
        \cCore=(\cAddr, \cSweep, \cDrift, \cKind). %, \cVisSw).
    \end{align}
%(where the field \( \cVisSw \) will also be explained later).

%     \item $\Af$ field of a cell will be set to true if
%     this cell belongs to a colony that ``represents'' a cell
%     that wants to become ``vacant''.
%     (We will explain  this notion later.)


%     The fields used in recovery mode are all
%     collected as subfields of the field
%     $\Rec$ of the state, and the field $\cRec$ of the cell state.
%     They will be introduced in the definition of the recovery rule.

%     In particular, when the field
%      \begin{align}\label{eq:cRecCore}
%       \cRec.\Core
%      \end{align}
%     is not $0$, we will call the cell \df{marked for recovery}.

%     Similarly, the fields used in rebuild mode are all
%     gathered as subfields of the field $\Arb$ of the state,
%     and the field $\cArb$ of the cell state.
%     We will use them in the definition of the rebuild rule.

%     Finally, when the field
%     In particular, when the field
%      \begin{align}\label{eq:cArbCore}
%       \cArb.\Core
%      \end{align}
%     is not $0$, we will call the cell \df{marked for rebuild}.

%     \item\label{i:fields.zigging}
%     The sweep-through is interrupted by switchbacks
%     called \df{zigging}.
%     While in the normal mode, the process depends
%     on two fixed parameters
%     \begin{equation}\label{eq:Z}
%      \begin{aligned}
%       &\Z \\
%       &\Z_b, \;\;\; \Z_b<\Z,
%      \end{aligned}
%      \end{equation}
%     that we will define later.
% %    where $\beta$ is the bound on the size of bursts.
%     This process is controlled by the fields $\ZigDepth$ and $\ZigDir$.

%     We assume that $Q$ is a multiple of $\Z - \Z_b$, that is
%     \begin{align}\label{eq:constraint-on-Z}
%         \Z - \Z_b \; | \; Q.
%     \end{align}
%     Every $\Z-\Z_b$ forward steps are accompanied by $\Z$ steps
%     backward and forward, for a total of  $3\Z-\Z_b$
%     steps.



\section{Error-correcting configuration code}\label{sec:coding}

Let us add error-correcting features to block codes introduced in
Definition~\ref{def:codes}.

\begin{definition}[Error-correcting code]\label{def:err-code}
A block code is $(\beta,t)$-\df{burst-error-correcting},
if for all $x\in\Sigma_{2}$, $y\in\Sigma_{1}^{Q}$ we
have $\psi^{*}(y)=x$ whenever $y$ differs from
$\psi_{*}(x)$ in at most $t$ intervals of size $\le\beta$.
\end{definition}

\begin{example}[Repetition code]\label{xmp:tripling}
  Suppose that $Q\ge 3\beta$ is divisible by 3,
  $\Sigma_{2}=\Sigma_{1}^{Q/3}$, $\psi_{*}(x)=xxx$.
  Let $\psi^{*}(y)$ be obtained as follows.
  If $y=y(1)\dots y(Q)$, then $x=\psi^{*}(y)$ is defined as follows:
    $x(i)=\maj(y(i),y(i+Q/3),y+2Q/3)$.
    For all $\beta\le Q/3$, this is a
    $(\beta,1)$-burst-error-correcting code.

    If we repeat 5 times instead of 3, we get a $(\beta,2)$-burst-error-correcting
    code.
    Let us note that there are much more efficient such codes than just repetition.
 \end{example}

Let us assume that a generalized Turing machine
\begin{align*}
    M = (\Gamma, \Sigma, \delta, \G, \cDir, q_{\start},F, \B, \Tu)
\end{align*}
is used to simulate a generalized Turing machine
\begin{align*}
M^* = (\Gamma^{*}, \Sigma^{*}, \delta^{*}, \G^{*}, \cDir^{*}, q^{*}_{\start},F^{*}, \B^{*}, \Tus).
\end{align*}
We will assume that $\Gamma^*\cup\set{\emptyset}$,
and the alphabet $\Sigma^*$ are subsets of the set of  binary strings
$\{0,1\}^{\ell}$ for some $\ell<Q$ (we can always ignore some states or tape
symbols, if we want).

We will store the coded information at distance 
\begin{align}\label{eq:penLen}
    \PadLen
\end{align}
from the end of our colony of size \( Q \).
So let $(\upsilon_{*}, \upsilon^{*})$ be a $(\beta,2)$-burst-error-correcting block code
\begin{align*}
  \upsilon_{*}: \{0,1\}^{\ell} \cup \set{\emptyset}
   \to\{0,1\}^{(Q-2\cdot\PadLen)\B}.
\end{align*}
We could use, for example, the repetition code of Example~\ref{xmp:tripling}.
Other codes are also appropriate, but we require that they have some fixed
programs $p_{\encode}$, $p_{\decode}$
on the universal machine $\Un$, in the following sense:
 \begin{align*}
   \upsilon_{*}(x)=\Un(p_{\encode},x),\quad
   \upsilon^{*}(y)=\Un(p_{\decode}, y).
 \end{align*}
Also, these programs must work in quadratic time and
linear space on a one-tape Turing machine (as the repetition code certainly does).

Let us now define the block code $(\psi_*, \psi^*)$ used in the
definition of the configuration code $(\phi_*, \phi^*)$ as 
outlined in Section~\ref{sec:hier-codes}.
We define
\begin{equation}\label{eq:psi}
   \psi_*(a)  = 0^{\PadLen}\upsilon_{*}(a)0^{\PadLen}.
\end{equation}
The decoded value \( \psi^{*}(x) \) is obtained by first removing \( \PadLen \) symbols
from both ends of \( x \) to get \( x' \), and then computing \( \upsilon^{*}(x') \).

To compute the configuration code \( \phi_{*} \) from \( \psi_{*} \) we proceed first as
done in Section~\ref{sec:hier-codes}, and then add the following initializations:
In cells of the base colony and its left neighbor  colony,
the $\cSweep$ and $\cDrift$ fields are set 
to $\Last(+1)-1$,  $1$, and $\Last(+1)$,  $1$ respectively.
In the right neighbor colony, these values are $\Last(-1)$ and $-1$ respectively.
In all other cells, these values are empty.
The $\cAddr$ fields of each colony are filled properly:
the $\cAddr$ of the $j$th cell of a colony
is $j \bmod \B^*$.

Decoding of configurations will be defined later, 
when defining the map \( \Phi^{*} \) of the simulation.
It is more complex since 
the location of the base colony must be found also in
configurations that are not in the range of the encoding
function, but arise as a result of a lot of noise.


\section{Simulation structure and coordination}

The first task of each simulation step is to check if the state
of the machine and the current cell are in a certain relationship.
When this condition is breached (due to some burst), machine calls
$\Alarm$ --- a rule that will initiate the recovery.
Below, we will specify these conditions, and
aggregate them into one, that we call the \df{coordination}.


The sketch of the main rule of machine $M$ is given in
Rule~\ref{alg:main1}, where the $\Comp$ and $\Transfer$ rules will be
outlined below.

\begin{algorule}[!ht] \caption{\textbf{Main rule}\label{alg:main1}}
  \If{\( \Mode=\Normal \)}{
    \lIf{\algNot \( \Coordinated \)}{\( \Alarm \)}\;
    \lElseIf{\( 1 \le \Sweep < \TransferSw(1) \)}{ \( \Comp \)
    }\;
    \lElseIf{\( \TransferSw(1) \le \Sweep < \Last \)}{ \( \Transfer \) 
    }\;
   \lElseIf{\( \Last \le \Sweep  \)}{move the head to the new base.}
  }
\end{algorule}


\section{Sweep counter and direction}\label{sec:sweep}

The global sweeping movement of the head will be
controlled by the parametrized rule
\begin{align*}
\ruSwing(a,b,u,v).
\end{align*}
This rule makes the head swing between two extreme points $a,b$,
while the counter $\Sweep$ increases from value $u$ to value $v$.
The $\Sweep$ value is incremented at the ``turns'' $a,b$ (and is
also recorded on the track $\cSweep$).

%\subsection{Yarding}%\label{sec:yarding}
%
%    For reasons that we will explain later, at certain sweeps,
%    the head will step outside of the current colony for certain
%    number of cells.
%    This property of our machine will be called the \df{yarding} property.
%
%    If when the head steps outside it does not find a cell,
%    it will create one.
%    While the head is on the cells that are outside,
%    it will save the current $\Sweep$ value
%    in the $\cVisSw$ and the current $\Addr$ modulo $Q$ in
%    $\cVisAddr$.
%
%    The range of penetration of the head in the right (resp., the left)
%    neighbor colony for each sweep that is greater than 1
%    is the number of trailing zeroes in a
%    binary notation of the $\Sweep - 1$ (resp., $\Sweep$), multiplied
%    by $\Z-4\beta$.


    The sweep direction $\delta$ of the simulating head is derived from
    $\Sweep$, $\Addr$ and the current value \( \Dir \) in the following way.
    On arrival of the head to an endpoint (that is
    when $\Dir \neq 0$ and $\Addr\in\{a,b\}$), the values
    $\Sweep$ and $\cSweep$ are incremented and $\Dir$ is set to 0.
    In all other cases, the sweep direction is determined
    using the formula
     \begin{align}\label{eq:sweep-dir}
       \dir(s)=(-1)^{s + 1}
     \end{align}
as follows: 
    \begin{equation}\label{eq:Dir}
    \delta =
      \begin{cases}
         0 & \text{ if $\Addr\in\{a,b\}$ and $\Dir\neq 0$},  \\
           \dir(\Sweep) &\text{otherwise}.
         \end{cases}
    \end{equation}
As we mentioned, each sweep will be broken up into zigzags to
allow the detection of premature turn-back.
At each non-zigging step, \( \Addr \gets \Addr+\delta \).

As an example of rules, we present the the zigging rule.
It uses the following parameters:

\begin{definition}\label{def:Z}
Let \( \Z=22\beta \), \( \Z_{b}=4\beta \).  %?
\end{definition}

Certain bursts may turn back the head prematurely,
causing a skip in the simulation.
We want to prevent this, since we 
would like the size of the structural repairs to be comparable to the burst size.
Zigging is introduced for this purpose.
Rule~\ref{alg:Zigzag}, itself uses the rule \( \Move(d) \), which 
we will defined later.  %?
A characteristic of this rule is that it changes nothing on the tape:
all its control information is in the state.
The rule repeats the following cycle: first it moves forward \( \Z-\Z_{b} \),
moving ahead the \df{front}, and performing all necessary operations
of the computation.
Then it moves backward and forward by \( \Z \) steps, not changing anything
on the tape (but as we will see, some consistencies will be checked).
The field \( \ZigDepth \) of the state measures the distance from the front
during the switchback.
From now on, we will assume that \( Q \) is a multiple of \( \Z-4\beta \).

% \begin{algorule}[!ht]\caption{\( \Move(d) \)}\label{alg:Move}

%   \( \Dir\gets d \)\Comment*[f]{\( d\in\{-1,0,1\} \).}\;

%   \lIf{\( \Mode=\Normal \)}{\( \Addr\gets\cAddr\gets\Addr+d \)}\;
%   \lElse{\( \Rec.\Addr\gets\Rec.\Addr+d \)}\;
%   Move in direction \( d \).
% \end{algorule}


\begin{algorule}[!ht]\caption{\( \Zigzag(d) \)}\label{alg:Zigzag}
  \Comment{\( d\in\{-1,1\} \) is the direction of progress.}
  \If{\( \ZigDir=-1 \) \algAnd ((\( \ZigDepth=0 \) \algAnd \( (\Z-4\beta)|\Addr \))
    \\ \algOr \( 0<\ZigDepth<\Z \))}{
    \( \increment{\ZigDepth} \)\;
    \lIf{\( \ZigDepth = \Z-1 \) %\algOr \( \cAddr\in\{-Q+1,2Q-2\} \)
    }{\( \ZigDir=1 \)}\;
    \( \Move(-d) \)
  }
  \ElseIf{\( \ZigDir=1 \) \algOr (\( \ZigDepth=0 \) \algAnd \( (\Z-4\beta)\not|\Addr \))}{
    \lIf{\( \ZigDepth>0 \)}{\( \decrement{\ZigDepth} \)}
    \lElse{\( \ZigDir\gets -1 \)}\;
    \( \Move(d) \)
  }
\end{algorule}

\section{Computation phase}\label{sec:computation-phase}

As shown in Rule~\ref{alg:main1} descibing a top-down view of the simulation,
the first phase of the simulation computes new values for state of the 
simulated machine \( M^{*} \)
represented on track $\cState$, the direction of the move of the head of  \( M^{*} \)
(represented in the $\cDrift$ field of each cell of the colony of \( M \)), and
the simulated cell state of \( M^{*} \) represented on the track $\cInfo$.
During this rule, the head sweeps the base colony.

The rule \( \Comp \) essentially repeats five times the following \df{stages}:
decoding, applying the transition, encoding, checking for destruction.

Due to the possibility of encountering a much larger burst of faults than
this rule can handle, some extra rules will then be applied that we will specify later:
$\UsefulComp$.  %?

In more detail:
\begin{enumerate}
     \item   In the first sweep of the work period, set
      $\cAddr \gets \cAddr\bmod{Q}$ and $\cKind \gets \Member$.

      \item For every $j=1,\dots,5$, if $\Addr \in \set{0, \dots, Q-1}$ do

       \begin{enumerate}

          \item Calling by $g$ the  string found on the $\cState$ track of
            the base colony between addresses $\PadLen$
            and $Q-\PadLen$, decode it into string $\tilde g=\upsilon^{*}(g)$
            (this should be the current state of the simulated machine), and
            store it on some auxiliary track in the base colony.
            Do this by simulating the universal machine on the $\cProg$ track,
            $\tilde g = \Un(p_{\decode}, g)$.

            Proceed similarly with the string $a$ found on the $\cInfo$
            track of the base colony, between addresses $\PadLen$
            and $Q-\PadLen$, to get $\tilde a = \upsilon^{*}(a)$
            (this should be the observed tape symbol of the simulated machine).

          \item \label{i:comp.trans}
           Compute the value \( (a',g',d)=\delta^{*}(\tilde a, \tilde g) \).
Since the neither the table nor any program of the transition 
function \( \delta^{*} \) is written explicitly anywhere, this ``self-simulation'' step needs
some elaboration, see Section~\ref{sec:self-simulation}.

            \item\label{i:comp.write}
              Write the encoded new state $\upsilon_{*}(g')$ onto the
              $\cHold[j].\State$ track of the base colony between positions
              $\PadLen\B$ and $Q-\PadLen\B$.
              Similarly, write the encoded new observed cell
              content $\upsilon_{*}(a')$ onto the $\cHold[j].\Info$ track.
              Write $d$ into the $\cHold[j].\Drift$ field of \emph{each cell} of
              the base colony.

              Special action needs to be taken in case 
              the new state \( g' \) is a vacant one, that is 
              \( g'.\Kind^{*}=\Vacant^{*} \).
              In this case, write \( 1 \) onto the \( \cHold[j].\Doomed \) track (else 0).

        \end{enumerate}

      \item Repeat the following twice (hoping that at least
        one repetition will be burst-free):

          \begin{quote}
            Sweeping through the base colony,
            at each cell compute the majority of \( \cHold[j].\Info \), \( j=1,\dots,5 \),
            and write into the field \( \cInfo \).
            Proceed similarly, and simultaneously, with \( \State \) and \( \Drift \).
          \end{quote}

       \item Call the rule $\UsefulComp$ that we will specify later.
  \end{enumerate}

It can be arranged---and we assume so---that the total number of sweeps of this
phase, and thus the starting sweep number of the next phase,
depends only on \( Q \).

\section{Self-simulation}\label{sec:self-simulation}

In describing the rule of the computation phase,
in the step~\ref{i:comp.trans} of Section~\ref{sec:computation-phase},
we said that machine $M$ writes the code \( p^{*} \)
of \( M^{*} \) onto the $\cProg$ track, without saying how this is done.
Here we give the details.

\subsection{New primitives}

We will make use of a special track
\begin{align*}
   \cWork
 \end{align*}
of the cells and the special field
\begin{align*}
   \Index
 \end{align*}
of the state of machine \( M \) that can store a certain address of a colony.

Recall from Section~\ref{sec:language} that the program
of our machine is a list of nested
``\textbf{if} \emph{condition} \textbf{then} \emph{instruction}
\textbf{else} \emph{instruction}''
statements.
As such, it can be represented as a binary string 
 \begin{align*}
   R.
 \end{align*}
If one writes out all details of the construction of the present paper, this string \( R \)
becomes completely explicit, an absolute constant.
But in the reasoning below, we treat it as a parameter.

There is a couple of \df{extra primitives} in the rules.
First, they have access to the parameter \( k \) of machine \( M=M_{k} \), 
to define the transition function
 \begin{align*}
            \delta_{R,k}(a,q).
 \end{align*}
The other, more important, new primitive is a special instruction
 \begin{align*}
   \WriteRulesBit
 \end{align*}
in the rules.
When called, this instruction makes the assignment \( \cWork\gets R(\Index) \).
This is the key to self-simulation: \emph{the program has
access to its own bits}.
If \( \Index=i \) then it writes \( R(i) \) onto the current position of the \( \cWork \) track.


\subsection{Simulating the rules}

By convention, in our fixed flexible universal machine \( \Un \),
program \( p \) and input \( x \) produce an output \( \Un(p,x) \).
Since the structure of all rules is very simple, they can be read and
interpreted by \( \Un \) in reasonable time:

\begin{theorem}
There is a constant string called \( \Interpr \) with the property that for
all positive integers \( k \), strings \( R \), \( a,q \) where \( R \) is a
sequence of rules, and bit strings \( a\in\Sigma_{k} \), \( q\in \Gamma_{k} \):
 \begin{align*}
  \Un(\Interpr,R,0^{k},a,q)=\delta_{R,k}(a,q).
 \end{align*}
The computation on \( \Un \) takes time \( O(|R|\cdot (|a|+|q|)) \).
\end{theorem}

The proof parses and implements the rules in the string \( R \); each of these rules
checks and writes a constant number of fields.

Implementing the \( \WriteRulesBit \) instruction is straightforward:
Machine \( \Un \) determines the number \( i \)
represented by the simulated \( \Index \) field, 
looks up \( R(i) \) in \( R \), and writes it into the simulated \( \cWork \) field.

Note that there is no circularity in these definitions:
  \begin{itemize}
  \item 
The instruction \( \WriteRulesBit \) is written \emph{literally}
in \( R \) in the appropriate place, as ``\(\WriteRulesBit \)''.
The string \( R \) is \emph{not part} of the rules (that is of itself).  
  \item On the other hand, the computation in
\( \Un(\Interpr,R,0^{k},a, q) \) 
has \emph{explicit} access to the string \( R \) as one of the inputs.
  \end{itemize}

Let us show the computation step invoking the ``self-simulation'' in detail.
In the earlier outline, step~\ref{i:comp.trans} of Section~\ref{sec:computation-phase},
just said to compute \( \delta^{*}(\tilde a, \tilde g) \)
(for the present discussion, we will just consider computing 
\( \delta^{*}(a,q)=\delta_{k+1}(a,q) \)), where \( \delta=\delta_{k} \),
where it is assumed that \( a \) and \( q \) are available on two appropriate
auxiliary tracks.
We give more detail now of how to implement this step:

\begin{enumerate}[1.]
\item Onto the \( \cWork \) track, write the string \( R \).
To do this, for \( \Index \) running from 1 to \( |R| \), 
execute the instruction \( \WriteRulesBit \) and move right.
Now, on the \( \cWork  \) track, replace it with \( \ang{\Interpr,0^{k+1},R,a,q} \).
Here, string \( \Interpr \) is a constant, so it is just hardwired.
String \( R \) already has been made available.
String \( 0^{k+1} \) can be written since the parameter \( k \) is available.
Strings \( a,q \) are available on the tracks where they were stored.
\begin{sloppypar}
 \item Simulate the universal automaton \( \Un \) on track \( \cWork \):
   it computes \( \delta_{R,k+1}(a,q)=\Un(\Interpr,R,0^{k+1}, a,q) \)
as needed.  
\end{sloppypar}
\end{enumerate}

This achieves the forced simulation.
Note what we achieved:

\begin{itemize}
  \begin{sloppypar}
\item On level 1, the transition function \( \delta_{R,1}(a,q) \) is defined completely
when the rule string \( R \) is given.
It has the forced simulation property by definition, and
string \( R \) is \emph{``hard-wired''} into it in the following way.
If \( (a',q',d)=\delta_{R,1}(a,q) \), then
\begin{align*}
  a'.\cWork=R(q.\Index)
\end{align*}
whenever \( q.\Index \) represents a number between 1 and \( |R| \).
and the values \( q.\Sweep \), \( q.\Addr \) satisfy the conditions
under which the instruction \( \WriteRulesBit \) is 
called in the rules (written in \( R \)).
      \end{sloppypar}

      \begin{sloppypar}
\item The forced simulation property of the \emph{simulated}
transition function \( \delta_{R,k+1}(\cdot,\cdot) \) is 
achieved by the above defined computation 
step---which \emph{relies on} the forced simulation property of \( \delta_{R,k}(\cdot,\cdot) \).
              \end{sloppypar}
\end{itemize}

\begin{remark}
This construction resembles the proof of Kleene's fixed-point theorem.
\end{remark}



\section{Transfer phase}\label{sec:TransferPhase}

If the simulated machine head moves left or right, then another phase will
be added to the simulation: transferring the simulated state information to the
neighbor colony, and to move there.
Let us call the direction of the transfer the \df{drift}.

During this phase, the range of the head
includes the base colony and the neighbor colony
determined by the drift, including a possible ``bridge'' between them.

The sweep in which we start transferring in direction \( \delta \) is called
\( \TransferSw(\delta) \), the \df{transfer sweep}.
We have \( \TransferSw(-1) =\TransferSw(1)+1 \).

\subsection{General structure of the phase}\label{sec:TransferPhase.general-struc}

We will make use of some extra rules that we will
specify in more detail later, but whose role is spelled out here.

The phase consists of the following actions.
\begin{enumerate}[1.]
\item
  Spread the value \( \delta \) found in the cells of the \( \cDrift \) track
  (they should all be the same)
  onto the neighbor colony in direction \( \delta \).

There are some details to handle in case the neighbor colony is not adjacent:
see Section~\ref{sec:adjacency}.

\item\label{i:transfer-state} For \( i=1,2,3 \):
        \begin{quote}
          Copy the content of \( \cState \) track of the base colony
            to the \( \cHold[i].\State \) track of the neighbor colony.
        \end{quote}

\item Repeat the following twice:
  \begin{quote}
 Assign the field majority: \( \cState\gets \maj(\cHold[1 \dots  3].\State) \)
in all cells of the neighbor colony.    
  \end{quote}
  
\item If \( \Drift = 1 \), then move right to the left endcell of the neighbor colony
(else you are already there).

        \begin{sloppypar}
          \item In the last sweep (possibly identical with the move step above), in the base colony,
            if the majority of \( \cHold[j].\Doomed \), \( j=1,\dots,5 \), is 1 then
            turn the scanned cell into a stem cell, setting \( \Kind\gets\Stem \): in other
            words, carry out the destruction.            
          \end{sloppypar}


\end{enumerate}

\subsection{Transfer to a non-adjacent colony}\label{sec:adjacency}

Let us address the situation when the neighbor colony is not adjacent.

\begin{definition}[Adjacency of cells]\label{def:adjacent}
  Cells $a$ and $b$ are \df{adjacent} if $|a-b|=\B$.
  Otherwise, if $\B < | a- b| < 2\B$, then
  $a$ and $b$ are two \df{non-adjacent neighbor cells}.
For the sake of the present discussion, a \df{colony} is a sequence of \( Q \) adjacent
cells whose \( \cAddr \) value runs from \( 0 \) to \( Q-1 \).

If the bodies of two cells are not adjacent, but are at a distance \( <\B \) then the space
between them is called a \df{small gap}.
We also call a small gap such a space between the bodies of two colonies.
On the other hand, if the distance of the bodies of two colonies is \( >\B \) 
but \( <Q\B \) then the space between them is called a \df{large gap}.
\end{definition}

In the transfer phase, in order to know in a robust local way where the head is,
the \( \cKind \) field of the cells visited will be set as follows.
The base colony has cells of kind \( \Member \) to begin with.
The kind of the cells of the neighbor colony, the target of the transfer, will be set
as \( \Neighbor \) for the duration of the transfer.
If there is a large gap between the base and the neighbor colony then it will be
filled with cells of the kind \( \Bridge \), forming a contiguous area of adjacent cells
with the base colony.
The bridge has its own addresses, starting from 0 if it extends to the right,
and from \( Q-1 \) if it extends to the left.

If while forming a bridge, an existing, non-adjacent 
cell of kind \( \Bridge \) or \( \Stem \)
is encountered, then this cell will be killed before an adjacent bridge cell is created.
(We will show later the rule implementing such a step.) %?
This process is called \df{realignment}.

If after forming a bridge,
it turns out that no neighbor colony exists (the bridge grows to length \( Q\B \)), 
we turn the bridge into an adjacent neighbor colony on the way back.

    Here are the details.

    \begin{flushdescription}
    \item [Assumptions]
      During the first sweep of transfer, in direction \( \delta \), 
      when we step onto a neighbor cell, this cell must either one of the following:
      (1) a stem cell (2) a member of a neighbor colony with or a cell of an old bridge: in
both cases, it must have $\cSweep = \Last(-\delta)$ and $\cDrift=-\delta$.
      Otherwise, $\Alarm$ will be called.

     \item [Extending a bridge]
       Here is how a bridge is extended from the base colony, in 
transfer direction \( \delta \).
Recall that the field \( \Kind \) of the state shows the kind of the cell visited
before the current one, whild \( \cKind \) is the kind of the cell being visited right now.

\begin{enumerate}[1.]

\item
  \begin{sloppypar}
 (Active part.)
  Suppose that $\cKind \in \{\Member,\Bridge\}$, \( \G=\true \).
Suppose further that either \( \Sweep=\cSweep+1 \) 
and \( \Kind\in\{\Member,\Bridge\} \) (the usual case)
or \( \Sweep=\cSweep \) and $\Kind  = \Vacant$.
(The case when the head presumably just returned from the cell it killed.
Normally \( \Sweep=\cSweep \) would trigger alarm, but not now.)    
  \end{sloppypar}

 Then, we move in direction $\delta$, with \( \Kind=\cKind \).

\item(Passive part.)
 Suppose that 1) \( \cKind =\Bridge \) and \( \cSweep=\Last(-\delta) \),
or \( \cKind=\Stem \), and 2) $\Kind  \in \set{\Member, \Bridge}$.

 \begin{enumerate}[a.]
 \item Suppose $\G = \false$.
Then $\cKind \gets \Bridge$, \( \cAddr\gets\Addr+\delta \).

If moving in direction \( \delta \) does not get outside the colony determined by \( \cAddr \)
then do it.
Else switch back onto a next sweep, and start turning the colony of \( Q \) 
bridge cells just created cells of kind \( \Neighbor \).

 \item Suppose \( \G=\true \).
Then kill this cell (that is $\cKind,\Kind \gets \Vacant$), 
and move in direction $-\delta$.
(Thus, we kill a non-adjacent bridge or stem cell and move back (in order 
to create an adjacent bridge cell).
\end{enumerate}

\item (The neighbor colony is reached.) 
  Suppose that $\Kind \in \{\Member,\Bridge \}$, $\cKind = \Member$, 
  $\cAddr = 0$ in case \( \delta=1 \) and $\cAddr =  Q-1$ in case \( \delta=-1 \).
  Then $\cKind, \Kind \gets \Neighbor$, $\Addr \gets \cAddr$.

  \end{enumerate}

  \item[Creating a neighbor colony]
    Suppose that $\cKind = \Member$, $\Kind  =\Neighbor$.
Let $\cKind \gets \Neighbor$.

    \begin{enumerate}[1.]
    \item Suppose that we have not reached the end of the first transfer sweep: that is 
if \( \delta=1 \) then \( \Addr<Q-1 \) (this is the address from where we are coming), 
and if \( \delta=-1 \) then \( \Addr>0 \).
Then move on in direction \( \delta \).
    \item Otherwise switch back, with \( \Sweep\gets\Sweep+1 \).
    \end{enumerate}

  \item[The rest of the transfer phase]
Once the bridge and the neighbor colony have been marked out the transfer phase
works as outlined at the beginning of the section.
But to pass between the base colony and the neighbor colony, the head always travels
through the bridge.

    \end{flushdescription}

Recall that the $\G$ field of the state determines
if the current cell is not adjacent to the cell  where the head came from.
After the transfer stage, we update the $\G^{*}$ field encoded in the
$\cState$ track of the neighbor colony: it becomes 1 if either there is a nonempty bridge,
or there is a gap (found with the help of the \( \G \) field) between the base colony
and the neighbor colony.

This is done in part~\ref{i:transfer-state} 
of Section~\ref{sec:TransferPhase.general-struc}
again three times, storing candidate values into $\cHold[j].\G$
and repeated with everything else.

 \subsection{Notes on zigging}\label{sec:zigging}

The zigging rule was introduced in Section~\ref{sec:sweep}.
We add now the following refinements to it.

When the head steps outside of the extended base colony and
does not find a cell, then it creates a stem cell whose
$\cDrift$ is set to $-\Dir$, where $\Dir$ is the field
of the state storing the direction of the last step (see~\ref{sec:fields}).
While zigging outside the extended base colony in normal mode,
we allow one small gap, next to the extended base colony
(other situations cause alarm).

% Later, in our analysis,
% we will need to estimate the number of steps needed
% to sweep a certain interval of cells where zigging is used.
% For this purpose we introduce  the
% function
% \begin{align}\label{eq:Zg}
%    \Zg(a,\Z, \Z_b),
% \end{align}
% where $a$ is the length of the interval, that is divisible by $\Z-\Z_b$.
% For example, $\Zg(Q,\Z, \Z_b)$ bounds the number of steps
% needed to sweep a colony with a zigging that
% occurs every $\Z - \Z_b$ steps.
% A simple bound on the values is
% \begin{align}\label{eq:Zg-bound}
%    \Zg(a,\Z, \Z_b) \leq \frac{a}{\Z - \Z_b}(3\Z - \Z_b).
% \end{align}



\bibliographystyle{plain}
\bibliography{reli,publ}

\end{document}
%%% Local Variables: 
%%% mode: latex
%%% TeX-master: t
%%% End: 
